%\documentclass[1p]{elsarticle}
\documentclass[11pt]{amsart}
\usepackage[mathscr]{eucal}
\usepackage{times}
\usepackage{amsmath,amssymb,latexsym,amscd}   
\usepackage{hyperref}
\hypersetup{colorlinks=true,linkcolor=blue,citecolor=brown,linktocpage=true}
%\input xypic
%\xyoption{all}
\usepackage{graphicx}
\usepackage[all,cmtip]{xy}
\usepackage{fancyhdr}
\usepackage{mathalfa}
\usepackage{mathrsfs}
%\usepackage[sans]{dsfont}
\usepackage{upgreek}
%\usepackage{mathpazo}
\usepackage{tikz}
\usepackage{graphicx}
\usepackage{tikz-cd}
%%%%%%%%%%%%%%%%%%

\DeclareMathOperator{\op}{op}
\DeclareMathOperator{\spec}{spec}
\DeclareMathOperator{\pt}{pt}
\DeclareMathOperator{\tp}{tp}
\DeclareMathOperator{\id}{id}
\DeclareMathOperator{\Sp}{sp}
\DeclareMathOperator{\Ord}{Ord}
\DeclareMathOperator{\Frm}{Frm}
\newcommand{\niton}{\not\owns}

\theoremstyle{plain}
\newtheorem*{theorem*}{Teorema}
\newtheorem{thm}{Teorema}[section]
\newtheorem{cor}[thm]{Corolario}
\newtheorem{lem}[thm]{Lema}
\newtheorem{prop}[thm]{Proposición}

\theoremstyle{definition}
\newtheorem{dfn}[thm]{Definición}
\newtheorem{obs}[thm]{Observación}
\newtheorem{ej}[thm]{Ejemplo}

\renewcommand{\proofname}{Demostración}
  %%
    %\providecommand{\corollaryname}{Corollary}
  %\providecommand{\examplename}{Example}
  %\providecommand{\lemmaname}{Lemma}
  %\providecommand{\propositionname}{Proposition}
  %\providecommand{\remarkname}{Remark}
%\providecommand{\theoremname}{Theorem}
%\providecommand{\subexamplename}{Subexample}
  %\providecommand{\definitionname}{Definition}
\begin{document}
%\begin{frontmatter}

\title{Compendio de resultados}


\author{}
\email{}
\address{}


\address{}
%$\cortext[cor1]{Corresponding author}

\begin{abstract}

\end{abstract}

\maketitle
\section{Preliminares}
En la categoría $\Frm$ los objetos son retículas completas que cumplen la ley distributiva 
de marcos (LDM) y los morfismos entre ellos son funciones monótonas que preservan la estructura de la retícula. Si $A\in \Frm$, entonces podemos asignarle un espacio topológico a 
través de $S=\pt A$, donde $p\in \pt A$ si $p$ es $\wedge$-irreducible.\\

Denotemos por $\mathcal{Q}S$ a la familia de todos los subconjuntos compactos saturados de $S$. Por el Teorema de Hoffman-Mislove (Teorema H-M), existe una correspondencia
biyectiva entre $\mathcal{Q}S$ y el premarco de los filtros abiertos de $A$ (denotado por $A^\wedge$). A su vez, si $F\in A^\wedge$, estos están en correspondencia biyectiva con lo que se conoce como 
núcleos ajustados y con ello obtenemos una versión extendida del teorema antes mencionado. En este caso, $v_F$ es el correspondiente núcleo ajustado asociado a $F$.\\

Si $j\in NA$, entonces $A_j=\{x\in A\mid j(x)=x\}$ es un marco y este es el marco cociente. El cociente $A_j$ es compacto si y solo si $\nabla(j)\in A^\wedge$, donde 
\[
\nabla(j)=\{x\in A\mid j(x)=1\}.
\]
En particular, $A_{v_F}$ es un cociente compacto.\\

Si $f^*\colon A\to B$ es un morfismo de marcos y $F\subseteq A$, $G\subseteq B$ son filtros en $A$, $B$, respectivamente, podemos producir nuevos filtros de la siguiente manera:
\begin{equation}\label{Imagenfiltros}
b\in f^*F \Leftrightarrow f_*(b)\in F\quad \mbox{ y }\quad a\in f_*G \Leftrightarrow f^*(a)\in G
\end{equation}
donde $a\in A, b\in B$ y $f_*$ es el adjunto derecho de $f^*$. Aquí $f^*F\subseteq B$ y $f_*G\subseteq A$ son filtros en $B$ y $A$, respectivamente.\\

\begin{prop}\label{fF}
Para $f=f^*\colon A\to B$ un morfismo de marcos y $G\in B^\wedge$, entonces $f_*G\in A^\wedge$.
\end{prop}

\begin{proof}
Por (\ref{Imagenfiltros}), $f_*G$ es un filtro en $A$. Necesitamos que $f_*G$ satisfaga la condición de filtro abierto. Sea $X\subseteq A$ tal que $\bigvee X\in f_*G$, con $X$ dirigido. Entonces
\[
Y=\{f(x)\mid x\in X\}
\] 
es dirigido y $f(\bigvee X)=\bigvee f[X]=\bigvee Y\in G$. Como $G$ es un filtro abierto, existe $y=f(x)\in Y$ tal que $y\in G$. Así $x\in f_*G$ de modo que $f_*G\in A^\wedge$.
\end{proof}

En \cite[Lema 8.9 y Corolario 8.10]{simmons2004vietoris} muestran que el diagrama
\[\begin{tikzcd}
	A & A \\
	{\mathcal{O}S} & {\mathcal{O}S}
	\arrow["{f^\infty}", from=1-1, to=1-2]
	\arrow["{U_A}"', from=1-1, to=2-1]
	\arrow["{U_A}", from=1-2, to=2-2]
	\arrow["{F^\infty}"', from=2-1, to=2-2]
\end{tikzcd}\]
conmuta laxamente, es decir, $U_A\circ f^\infty\leq F^\infty \circ U_A$. En este diagrama $U_A$ es el morfismo reflexión espacial, $f^\infty$ y $F^\infty$ representan los núcleos asociados a los filtros $F\in A^\wedge$ y $\nabla\in \mathcal{O}S^\wedge$. También $f^\infty$ y $F^\infty$ son las cerraduras idempotentes asociadas a 
a los prenúcleos $f$ y $F$ respectivamente.\\

Aquí probamos algo más general, ya que consideremos el diagrama
\[\begin{tikzcd}
	A & A \\
	{A_j} & {A_j}
	\arrow["{\hat{f}^\infty}", from=1-1, to=1-2]
	\arrow["j"', from=1-1, to=2-1]
	\arrow["j", from=1-2, to=2-2]
	\arrow["{f^\infty}"', from=2-1, to=2-2]
\end{tikzcd}\]
donde $\hat{f}^\infty$ es el núcleo asociado al filtro $j_*F\in A^\wedge$ y $j\in NA$.  

\begin{lem}\label{f1f}
    Para $j$, $f$ y $\hat{f}$ como antes se cumple que $j\circ \hat{f}\leq f\circ j$.
\end{lem}
\begin{proof}
    Por (\ref{Imagenfiltros})
    \[
    \hat{f}=\dot{\bigvee}\{v_y\mid y\in j_*F\}\quad  \mbox{ y } \quad f=\dot{\bigvee}\{v_{j(y)}\mid j(y)\in F\}. 
    \]
entonces para $a\in A$ se cumple
\[
v_y(a)=(y\succ a)\leq \hat{f}(a)\leq j(\hat{f}(a)).
\]

También, para todo $a, y\in A$, $(y\succ a)\wedge y=y\wedge a$ y
\[
\begin{split}
j((y\succ a)\wedge y)\leq j(a) & \Leftrightarrow j(y\succ a)\wedge j(y)\leq j(a)\\
& \Leftrightarrow j(y\succ a)\leq (j(y)\succ j(a)).
\end{split}
\]

Así 
\[
v_y(a)\leq j(\hat{f}(a))\leq (j(y)\succ j(a))=v_{j(y)}(j(a))\leq f(j(a)).
\]

Por lo tanto $j\circ \hat{f}\leq f\circ j$.
\end{proof}

Ahora probaremos lo anterior, pero para cualquier ordinal.

\begin{cor}\label{finftyf}
    Para $j$, $f$ y $\hat{f}$ como antes, se cumple que $j\circ \hat{f}^\alpha\leq f^\alpha\circ j$
\end{cor}

\begin{proof}
    Para $\alpha\in \Ord$ verificaremos que $j\circ \hat{f}^\alpha\leq f^\alpha\circ j$. Lo haremos con inducción transfinita.\\

    Si $\alpha=0$, es trivial.\\

    Para el paso de inducción, supongamos que para $\alpha$ se cumple. Luego
    \[
    j\circ \hat{f}^{\alpha+1}=j\circ \hat{f}\circ \hat{f}^{\alpha}\leq  f\circ j\circ \hat{f}^\alpha\leq f\circ f^\alpha\circ j=f^{\alpha+1}\circ j,
    \]
    donde la primera desigualdad es el Lema \ref{f1f} y la segunda es la hipótesis de inducción.\\

    Si $\lambda$ es un ordinal límite, entonces
	\[
	\hat{f}^\lambda=\bigvee\{\hat{f}^\alpha\mid \alpha<\lambda\}, \quad f^\lambda=\bigvee\{f^\alpha\mid \alpha<\lambda\} 
	\]
	y
	\[
		j\circ \hat{f}^\lambda=j\circ \bigvee_{\alpha<\lambda}\hat{f}^\alpha\leq\bigvee_{\alpha<\lambda}j\circ \hat{f}^\alpha.
	\]
	Por la hipótesis de inducción tenemos que 
	\[
	j\circ \hat{f}^\alpha\leq f^\alpha\circ j\Rightarrow \bigvee_{\alpha<\lambda}j\circ \hat{f}^\alpha\leq \bigvee_{\alpha<\lambda} f^\alpha\circ j.
	\]
	Por lo tanto $j\circ \hat{f}^\lambda\leq f^\lambda\circ j$.
\end{proof}

Por el Corolario \ref{finftyf}, $j\circ \hat{f}^\infty\leq f^\infty\circ j$ se cumple. Además, por el Teorema H-M, $f^\infty=v_F$ y $\hat{f}^\infty=v_{j_*F}$. 
Con esto en mente, tenemos el siguiente diagrama
\[\begin{tikzcd}
	A && {A_{j_*F}} \\
	\\
	{A_j} && {A_F}
	\arrow["{(v_{j_*F})^*}", shift left, from=1-1, to=1-3]
	\arrow["j"', from=1-1, to=3-1]
	\arrow["H"', from=1-1, to=3-3]
	\arrow["{(v_{j_*F})_*}", shift left, from=1-3, to=1-1]
	\arrow["{(v_F)^*}"', shift right, from=3-1, to=3-3]
	\arrow["{(v_F)_*}"', shift right, from=3-3, to=3-1]
\end{tikzcd}\]
donde $A_F$ y $A_{j_*F}$ son los cocientes compactos producidos por $v_F$ y $v_{j_*F}$, respectivamente. El morfismo $H\colon A\to A_F$ está definido por $H=v_F\circ j$. Además, $(v_F)_*$ y $(v_{j_*F})_*$ son inclusiones.\\

Sea $h\colon A_{j_*F}\to A_j$ tal que, para $x\in A_{j_*F}$, $h(x)=H(x)$. Así si $h=H_{\mid{A_{j_*F}}}$, el diagrama anterior conmuta.\\

Primero necesitamos que $h$ sea un morfismo de marcos. Por la definición de $h$, este es un $\wedge$-morfismo y restaría verificar que es un $\bigvee$-morfismo.\\

Los supremos en $A_{j_*F}$ y $A_F$ son calculados de manera diferente. De esta manera, sea $\hat{\bigvee}$ el supremo en $A_{j_*F}$ y $\tilde{\bigvee}$ el supremo en $A_F$. Por lo tanto
\[
\hat{\bigvee}=v_{j_*F}\circ \bigvee\quad\mbox{ and }\quad \tilde{\bigvee}=v_{F}\circ \bigvee,
\] 
es decir, para $X\subseteq A$, $Y\subseteq A_j$,
\[
	\hat{\bigvee}X=v_{j_*F}(\bigvee X)\quad\mbox{ and }\quad \tilde{\bigvee}Y=v_{F}(\bigvee Y).
\]

Como $H$ es un morfismo de marcos, $H\circ \bigvee=\tilde{\bigvee}\circ H$. Necesitamos algo similar para el morfismo $h$.

\begin{lem}\label{bigvee g}
$h\circ \hat{\bigvee}=\tilde{\bigvee}\circ h$.
\end{lem}

\begin{proof}
Basta con verificar la desigualdad $h\circ \hat{\bigvee}\leq \tilde{\bigvee}\circ h$. Notemos que
\[
h\circ \hat{\bigvee}=H\circ v_{j_*F}\circ \bigvee=v_F\circ j\circ v_{j_*F}\circ \bigvee\leq v_F\circ v_F\circ j\circ \bigvee
\]
donde la desigualdad es el Corolario \ref{finftyf}. Además, $v_F\circ v_F=v_F$ y
\[
h\circ\hat{\bigvee}\leq v_F\circ j\circ \bigvee =H\circ \bigvee=\tilde{\bigvee}\circ H=\tilde{\bigvee}\circ h.
\]
Por lo tanto $h\circ\hat{\bigvee}=\tilde{\bigvee}\circ h$.
\end{proof}

De esta manera podemos enunciar el siguiente resultado.

\begin{prop}\label{VFsquare}
El diagrama
\[\begin{tikzcd}
	A & {A_{j_*F}} \\
	{A_j} & {A_F}
	\arrow["{v_{j_*F}}", from=1-1, to=1-2]
	\arrow["j"', from=1-1, to=2-1]
	\arrow["h", from=1-2, to=2-2]
	\arrow["{v_F}"', from=2-1, to=2-2]
\end{tikzcd}\]
es conmutativo.
\end{prop}

\begin{proof}
HAY QUE PONER LA PRUEBA
\end{proof}

Con el diagrama anterior podríamos analizar algunos cocientes compactos, por ejemplo, los cocientes compactos cerrados.


\section{Marcos eficientes y marcos $KC$}

En \cite{sexton2006point} Sexton dice que $A\in \mathbf{Frm}$ es \emph{eficiente} si para todo $F\in A^\wedge$
\[
x\in F\Rightarrow u_d(x)=d\vee x=1
\]
donde $d=d(\alpha)=f^\alpha(0)$, $f=\dot{\bigvee}\{v_y\mid y\in F\}$ y $v_y$ es un $v$-núcleo. Queremos trasladar esta misma noción, pero para $A_j$ cuando $j\in NA$, es decir, para todo $F\in A_j^\wedge$, si $x\in F$ entonces $d\vee x=1$, con $d$ similar al antes, pero en este caso tomamos $v_y\in NA_j$ y $0_{A_j}=j(0)$.\\

\subsection{Algunas propiedades de los marcos eficientes}
Los marcos eficientes fueron introducidos como una especie de propiedad de separación dada en el lenguaje 
libre de puntos, pero que se relaciona con propiedades sensibles a puntos. Un ejemplo de lo anterior es el
siguiente.

\begin{cor}
    Para $A$ un marco espacial, $\mathcal{O}S$ es un marco Hausdorff si y solo si $A$ es $1-$arreglado.
\end{cor}

\begin{proof}
    Se sigue del hecho de que la propiedad Hausdorff es conservativa y por el Teorema 8.4.4 de \cite{sexton2006point}
\end{proof}

La eficiencia es una propiedad más fuerte que ser $T_1$. La pregunta natural que surge es ¿resulta ser más fuerte que $T_2$ (o las propiedades tipo Hausdorff en $\Frm$)?

%\begin{prop}
 %   Todo marco Hausdorff es arreglado.
%\end{prop}

%\begin{proof}
    %Consideramos $A$ un marco Hausdorff, entonces para cualquier $1\neq a\nleq b\in A$ existe $u\in A$ tal que $u\nleq a$ y $\neg u\nleq b$. Debemos verificar que para cada $F\in A^\wedge$
    %\[
    %\mbox{si } x\in F\Rightarrow d(1)\vee x=1,
    %\]
    %donde $d(1)=f(0)$ y 
    %\[
    %f(0)=\bigvee\{v_a(0)\mid a\in F\}=\bigvee\{\neg a\mid a\in F\}.
    %\]
    %Por contradicción, supongamos que $A$ no es $1-$arreglado, es decir, existe $F\in A^\wedge$ tal que $F$ no es $1-$arreglado. 

    %Sin perdida de generalidad, supongamos que $F\neq A.$ Notemos lo siguiente:
    %\begin{enumerate}
     %   \item $d(1)\neq 1$, pues de lo contrario $f(0)=1$. Así $f=\tp=f^\infty$ y por lo tanto $F=\nabla(\tp)=A$, lo cual descartamos antes.

      %  \item $x\nleq d(1)$, pues de lo contrario si $x\leq d(1)$, entonces $d(1)\in F$ y así, por la definición de $f$ se cumple que 
       % \[
        %v_{d(1)}\leq f\Leftrightarrow f(d(1))=1\Leftrightarrow f(f(0))=1\Leftrightarrow f^2(0)=1
        %\]
        %es decir, $f^\infty=\tp$ y es la situación descartada.

        %\item $x\neq 1$, pues si $x=1$, entonces $x\vee d(1)=1$, lo cual contradice la suposición.

    %\end{enumerate}
%lo dearriba ya estaba comentado antes de comentar lo que sigue
%    Por lo tanto tenemos que $1\neq x\nleq d(1)$. Así, aplicando $\mathbf{(H)}$ existe $u\in A$ tal que $u\nleq x$ y $\neg u\nleq d(1)$.\\
%
%    Notemos que $u\notin F$ ya que, en caso contrario, si $u\in F$, entonces $v_u\leq f$. Luego $v_u(0)=\neg u\leq f(0)$ lo cual contradice $\mathbf{(H)}$.\\

%    Por el Lema de separación de Birkhoff (Teorema 2.4 artículo de Rosy), existe un filtro completamente primo $G$ tal que $u\notin G\supseteq F$. Por lo correspondencia biyectiva entre filtros completamente primos y puntos en $A$, tenemos que $G=\nabla(w_p)$, donde $p=\bigvee\{y\mid y\notin G\}$ y $F\subseteq \nabla (w_p)$.\\
    
%    Como $u\notin G$, entonces $u\leq p$. Además, $A$ es $T_{1_S}$ y por lo tanto todo punto es máximo. Por el Lema 9.4.3 de la tesis de Rosy, $u_p=w_p$, es decir, $F\subseteq G=\nabla(u_p)$.\\

%    Por hipótesis, $x\in F$ y $1\neq x$. Luego $1\neq x\nleq p$, pues $x\in G$. Entonces por $\mathbf{(H)}$ existe $u\in A$ tal que
%    \[
%    u'\nleq x\quad\mbox{ y }\quad \neg u'\nleq p
%    \]

%esto también ya estaba comentada antes de comentar lo de arriba

%    \textbf{Afirmación:} Con las hipótesis anteriores, $x$ es máximo.\\

 %   Consideremos $y\in A$ tal que $x\leq y$. Debemos verificar que $x=y$ o $y=1$. Si $x\leq y$, entonces 
  %  \[
   % 1=x\vee p\leq y\vee p
    %\]
    %Lo cual implica que $y\vee p=1$. Luego, como $p\leq y\vee p$ y $p$ es máximo, se debe cumplir que $y\vee p=p$ o $y\vee p=1$.

%\begin{itemize}
 %   \item Si $y\vee p=p$, entonces $y\leq p$, pero como $x\leq y$ implica que $y\nleq p$. Por lo tanto $x=y$. 
  %  \item Si $y\vee p=1$, por la maximalidad de $p$ se cumple que $y=1$.
%\end{itemize}
%Por lo tanto $x$ es máximo.\\

%Notemos que $x$ y $p$ son máximos.\\

%\textbf{Afirmación:} Si $x$ y $p$ son máximos. Entonces 
%\[
%\forall u, v\in A \mbox{ se cumple que }u\leq x \mbox{ o }v\leq p \mbox{ o }u\wedge v\neq 0.
%\]

%Sean $u, v\in A$. Sin perdida de generalidad, consideremos $u,v\neq 0$, pues de manera trivial se cumplen dos de las condiciones anteriores. Por la maximalidad de $x$ y $p$ 

%\textbf{Comentarios:}

% lo de arriba ya estaba comentado antes de lo que sigue
%
%Sabemos que para $F\in A^\wedge$ y $d=d(\infty)=v_F(0)$ se cumple que $u_d\leq v_F$. Si $A$ es arreglado, entonces $u_d\geq v_F$.\\

%A manera de contradicción, supongamos que $A$ no es arreglado, es decir $u_d\neq v_F$. Si $F\in A^\wedge$, entonces $[v_F, w_F]$ es el respectivo intervalo de admisibilidad. Así $u_d\lneq w_F$ (o equivalentemente $w_F\nleq u_d$, es decir, existe $a\in A$ tal que $w_F(a)\nleq u_d(a)$. Notemos que 
%\[
%w_F(a)=\bigwedge\{p\in M\mid a\leq p\}\neq 1,
%\]
%así, por $\mathbf{(H)}$, existe $v\in A$ tal que $v\nleq w_F(a)$ y $\neg v\nleq u_d(a)$. Por monotonía, 
%\[
%w_F(0)\leq w_F(a)\quad \mbox{y}\quad u_d(0)\leq u_d(a).
%\]
%Así, $v\nleq w_F(0)$ y $\neg v\nleq d=f^\alpha(0)$, en particular $\neg v\nleq f(0)=\bigvee\neg a\mid a\in F\}.$ De esta manera, $v\notin F$, lo cual implica $v\in F'=A\setminus F$, entonces existe $m\in M$ tal que $v\leq m$. Además, $w_F(0)=\bigwedge M$, es decir, $v\nleq p$ para algún $p\in M$.
    
%\end{proof}

\begin{thm}
    Todo marco fuertemente Hausdorff es arreglado.
\end{thm}

\begin{proof}
    Consideremos $A\in \Frm$ fuertemente Hausdorff. Si $A$ cumple $\mathbf{(fH)}$, entonces todo sublocal compacto es cerrado. Por teoría de marcos, para $j\in NA$ arbitrario, $A_j$ es compacto si y solo si $\nabla(j)\in A^\wedge$. De aquí que, al ser compacto y por $\mathbf{(fH)}$ $A_j=A_{u_d}$, para algún $d\in A$, es decir, $j=u_d$ y $\nabla(j)=\nabla(u_d)$ para algún $d\in A$, en particular, por H-M, para todo $F\in A^\wedge$, $v_F\in NA$. Así $\nabla(v_F)=\nabla(u_d)$, es decir, para $x\in F$ se cumple que $u_d(x)=1=d\vee x$. Por lo tanto $A$ es arreglado.
\end{proof}

Es momento de verificar si la eficiencia es una propiedad que se preserva bajo cocientes.

\begin{prop}\label{tidyquout}
    Si $A$ es un marco eficiente, entonces $A_j$ es un marco eficiente.
\end{prop}

\begin{proof}
Es fácil verificar que $F\subseteq j_*F$. Como $A$ es eficiente y $F\in A^\wedge$, se cumple que 
\[
x\in F\Rightarrow \hat{d}\vee x=1,
\]
donde $\hat{d}=d(\alpha)=f^\alpha(0)$.\\

Si $\hat{d}\leq d$, entonces $d\vee x=1$, para $d=d(\alpha)=f^\alpha(j(0))$.\\

Así, por el Corolario \ref{finftyf}
\[
\hat{d}=\hat{d}(\alpha)\leq j(\hat{d}(\alpha))=j(\hat{f}^\alpha(0))\leq f^\alpha(j(0))=d(\alpha)=d.
\]

Por lo tanto, si $x\in F$, entonces $d\vee x=1$ y $A_j$ es eficiente.
\end{proof}

\begin{dfn}\label{KC y KCH}
Sea $A\in \Frm$ decimos que $A$ es:
\begin{enumerate}
    \item $KC$ si cada cociente compacto es cerrado.

    \item \emph{Hausdorff cerrado compacto} (o $KCH$ de manera abreviada) si cada cociente compacto de $A$ es cerrado y Hausdorff.
\end{enumerate}
\end{dfn}

En otras palabras, la Definición \ref{KC y KCH} nos dice que si $j\in NA$, entonces $\nabla(j)\in A^\wedge$ y $j=u_a$ para algún $a\in A$. De manera adicional, para $2)$ pedimos que $A_j$ cumpla la propiedad $\mathbf{(H)}$.\\

\subsection{Propiedades de los marcos $KC$}


\begin{prop}\label{KCquout}
    Si $A\in \Frm$ comple $KC$, entonces $A_j$ cumple $KC$ para cada $j\in N(A).$
\end{prop}

\begin{proof} %HAY que redactyar mejor la prueba%
Consideremos $k\in NA_j$ tal que $(A_j)_k$ es compacto. 
Como cualquier filtro abierto es admisible, tenemos que $\nabla(k)\in A_j^\wedge$ 
y por la Proposición \ref{fF} $j_*\nabla(K)\in A^\wedge$.\\

Sea $l=j_*\circ  k\circ j^*\in NA$, entonces $A_l$ es un cociente compacto de $A$ y existe $a\in A$ tal que $l=u_a$. Así
\[
\begin{tikzcd}
	A \arrow[r, "j^*"'] \arrow[rrr, "l", bend left] & A_j \arrow[r, "k"'] & (A_j)_k \arrow[r, "j_*"'] & A_j\subseteq A
	\end{tikzcd}\]
y $a\vee x=k(j(x))$. Por lo tanto, si $x=a$, $k(j(x))=a$.\\

Necesitamos que $k=u_b$ para algún $b\in A_j$. Para $x\in A_j$ y $b=j(a)$
\[
\begin{split}
u_b(x)= b\vee x= b\vee j(x)& =j(j(a)\vee j(x))\\
& =j(k(j(a))\vee x)\\
& =j(u_a(x))\\
& =j(k(x))\\	
&=k(x).
\end{split}
\]
Por lo tanto $u_b=k$.
\end{proof}

\begin{prop}\label{KCT1}
Si $A$ es un marco $KC$, entonces $A$ es un marco $T_1$.
\end{prop}

\begin{proof}
Sean $p\in \pt A$ y $a\in A$ tales que $p\leq a\leq 1$. Consideremos 
\[
w_p(x)=\left\{\begin{array}{lcc}
1 & \mbox{ si } & x\nleq p\\
\\
p & \mbox{ si } & x\leq p
\end{array}\right.
\]
para $x\in A$. $P=\nabla(w_p)=\{x\in A\mid x\nleq p\}$ es un filtro completamente primo (en particular, $P\in A^\wedge$). Como $A$ es $KC$, entonces $A_{w_p}$ es un cociente compacto cerrado. Así $u_p=w_p$ y
\[
u_p(a)=a\quad \mbox{and}\quad w_p(a)=1.
\]
es decir, $a=1$. Por lo tanto $p$ es máximo. 
\end{proof}

\begin{prop}\label{coprod}
Las siguientes afirmaciones son ciertas:
\begin{itemize}
\item[(1)] La clase de marcos eficientes es cerrada bajo coproductos.
\item[(2)] La clase de marcos $KC$ es cerrada bajo coproductos. 
\end{itemize}	
\end{prop}

\begin{proof}

\end{proof}

\section{Familias particulares de núcleos}
Nuestro objetivo es estudiar los núcleos que producen cociente compacto. De manera particular, estamos interesados en 
aquellos núcleos que producen cociente compacto y cerrado.\\

Lo presentado en esta sección es una generalización de lo hecho por Escardó en \cite{escardo2006compactly}. En este caso, el lugar de utilizar la propiedad
$\mathbf{(fH)}$, usamos la eficiencia (o $KC$ cuando sea necesario).

\begin{dfn}\label{Definicion2.1}
Para $A\in \Frm$ y $j\in NA$, decimos que $j$ es $kq$ (por cociente compacto) si $A_j$ es compacto.  
\end{dfn}

Denotamos por 
\[
\mathfrak{K}A=\{j\in NA\mid j \mbox{ es } kq\}.
\]

Sabemos que los $u$-núcleos producen cocientes cerrados. De esta manera, un $u$-núcleo produce cociente compacto si $u_{\bullet}\in \mathfrak{K}A$ para $\bullet\in A$. Consideremos los subconjuntos
\[
\mathfrak{C}A=\{u_\bullet\in NA\mid u_\bullet\in \mathfrak{K}A\}\quad \mbox{ y }\quad \mathfrak{c}A=\{c\in A\mid u_c\in \mathfrak{K}A\}.
\]
donde $\mathfrak{C}A\cong \mathfrak{c}A$. El uso de cada uno de ellos depende del enfoque que ocupemos, ya sea como elementos del marco principal o como núcleos.

Para $j, k\in NA$ tenemos la relación de equivalencia dada por 
\[
j\sim k \Leftrightarrow \nabla(j)=\nabla(k),
\]
donde $\nabla( \_ )$ es el filtro de admisibilidad correspondiente al respectivo núcleos.\\

La clase de cada núcleo define un bloque en $NA$ y se puede demostrar que cada uno de estos bloques tiene menor elemento. Al menor elemento del bloque se le conoce como núcleo ajustado y todo núcleo ajustado tiene la forma
\[
f=\dot\bigvee \{v_a\mid a\in F\}
\]
donde $F$ es un filtro en $A$ y $\dot{\bigvee}$ es el supremo puntual.

\begin{lem}\label{cn ajustado}
    Consideremos $j\in NA$ y supongamos que $j$ es ajustado. Entonces
    \[
    j\leq k\Leftrightarrow \nabla(j)\subseteq \nabla(k)
    \]
    para todo $k\in NA$.
\end{lem}

Si $F\in A^\wedge$, $F=\nabla(j)$ para algún $j$ en $NA$, el bloque de $j$ siempre tiene un mayor y un menor elemento. Denotamos al menor elemento del bloque por  $v_F=f^\infty$.\\

Por último, si el marco $A$ es eficiente, se cumple que $v_F=u_d$, donde $d=v_F(0)$. Notemos que la eficiencia proporciona un cociente compacto y cerrado (el respectivo $v_F$), 
para cada $F\in A^\wedge$.\\

Con todo lo anterior tenemos la siguiente relación entre los distintos núcleos mencionado hasta este momento
\[
\mathfrak{f}\mathfrak{C}A\subseteq \mathfrak{C}A\subseteq \mathfrak{K}A\subseteq NA
\]
donde $\mathfrak{f}\mathfrak{C}A$ es el conjunto de núcleos $v_F=u_d$ que se obtienen cuando un marco es eficiente. Además, con respecto a los marcos se cumple que
\[
KCH\Rightarrow KC\Rightarrow \mbox{Eficiente}.
\]

\begin{lem}\label{Lema2.2}
    Para $c\in A$ las siguientes son equivalentes:
    \begin{enumerate}
        \item $c\in \mathfrak{c}A$.
        \item $A_{u_c}$ es un marco compacto.
        \item $\nabla(u_c)\in A^\wedge$.
    \end{enumerate}
\end{lem}

\begin{proof}
    $1)\Rightarrow 2)$ se cumple por caracterización de marcos compactos. Si se cumple $3)$, entonces $c\in \mathfrak{c}A$. Por último, $2) \Leftrightarrow 3)$ es cierto por la caracterización de cocientes compactos y filtros admisibles. 
\end{proof}

\begin{lem}\label{Lema2.3}
    Lo siguiente se cumple:
    \begin{enumerate}
        \item $\mathfrak{c}A$ es una sección superior.
        \item Si $X\subseteq \mathfrak{c}A$, entonces $\bigvee X\in \mathfrak{c}A$ y si $c, c'\in \mathfrak{c}A$ entonces $(c\succ c')\in \mathfrak{c}A$.
        \item Si $c, c'\in \mathfrak{c}A$, entonces $c\wedge c'\in \mathfrak{c}A$.
        \item $\mathfrak{c}A\subseteq A$ es un premarco con implicación.
        \item $\mathfrak{c}A$ es un premarco compacto.
    \end{enumerate}
\end{lem}

\begin{proof}
    \begin{enumerate}
        \item Sean $c\leq c'$ tal que $c\in \mathfrak{c}A$, entonces $\nabla(u_c)\subseteq \nabla(u_{c'})$. Consideremos $X\subseteq A$ dirigido con $\bigvee X\in \nabla(u_{c'})$. Debemos probar que 
        \[
        \nabla(u_{c'})\cap X\neq \emptyset.
        \]
        
        Notemos que 
        \[
        c'\vee (\bigvee X)=1\Rightarrow c\vee (c'\vee (\bigvee X))=1\Rightarrow c'\vee (\bigvee X)\in \nabla(u_c). 
        \]
        y el conjunto $Y=\{c'\vee x\mid x\in X\}$ es dirigido tal que $\bigvee Y\in \nabla(u_c)$. Al ser $\nabla(u_c)$ abierto, se tiene que existe $y\in Y$ tal que $y\in \nabla(u_c)$ con $y=c'\vee x$. Además, $y\in \nabla(u_{c'})$, es decir,
        \[
        y\vee c'=(c'\vee x)\vee c'= c'\vee x=1
        \]
        Por lo tanto $x\in \nabla(u_{c'})\in A^\wedge$ y así $c'\in \mathfrak{c}A$.
        \item Se cumple por ser sección superior.
        \item Si $c, c'\in \mathfrak{c}A$, entonces $u_c, u_{c'}\in \mathfrak{K}A$. De esta manera, debemos verificar que $u_{c\wedge c'}\in \mathfrak{K}A$. Por propiedades de los $u$-núcleos $u_{c\wedge c'}=u_c\wedge u_{c'}$, entonces
        \[
            \nabla(u_{c\wedge c'})=\nabla(u_c\wedge u_{c'})=\nabla(u_c)\cap \nabla(u_{c'}).
        \]
        Consideremos $X\subseteq A$ dirigido tal que $\bigvee X\in \nabla(u_{c\wedge c'})$. De aquí que $\bigvee X\in \nabla(u_c)\cap \nabla(u_{c'})$. Como ambos son filtros abiertos, existe $x\in X$ tal que $x\in \nabla(u_c)$ y $x\in \nabla(u_{c'})$, es decir, $x\in \nabla(u_c)\cap \nabla(u_{c'})$. Por lo tanto $\nabla(u_{c\wedge c'})\in A^\wedge$, es decir, $c\wedge c'\in \mathfrak{C}A$.
        \item Es consecuencia de $1)$, $2)$ y $3)$.
        \item Sea $X$ un conjunto dirigido de elementos en $\mathfrak{c}A$ tal que $\bigvee X=1$ y consideremos $c\in \mathfrak{c}A$. Como $u_c$ es $kq$, entonces $A_{u_c}$ es compacto y así $1$ es compacto en $A_{u_c}$, es decir, para
        \[
        \bigvee X=\bigvee^{u_c}X=1
        \]
        existe $x\in X$ tal que $x=1$ y $1\in A_{u_c}$.  Por lo tanto, $\mathfrak{c}A$ es compacto.
    \end{enumerate}
\end{proof}

\begin{lem}\label{Lema2.8y2.9}
    Sea $A\in \Frm$ y $\alpha_A\colon \mathfrak{c}A\to A^\wedge$ la función definida por
    \[
    \alpha_A(c)=\{x\in A\mid c\vee x=1\}=\nabla(u_c).
    \]
    Lo siguiente se cumple:
    \begin{enumerate}
        \item $\alpha_A$ es un morfismo de premarcos.
        \item Si $A$ es eficiente, $\alpha_A$ es sobreyectiva.
    \end{enumerate}
\end{lem}

\begin{proof}
    \begin{enumerate}
        \item Veamos primero que $\alpha_A$ preserva la estructura de un premarco.
    \begin{enumerate}
        \item Si $c\leq c'$,  entonces $\alpha_A(c)=\nabla(u_c)\subseteq \alpha_A(c')=\nabla(u_{c'})$.
        \item Consideremos $1\in \mathfrak{c}A$, entonces 
        \[
        \alpha_A(1)=\nabla(\tp)=A,
        \]
        de aquí que, $\alpha_A$ preserva el mayor elemento.
        \item Sean $c_1, c_2\in \mathfrak{c}A$ y $X\subseteq \mathfrak{c}A$ dirigido. De aquí que
        \[
        \begin{split}
            \alpha_A(c_1\wedge c_2)&=\{x\in A\mid (c_1\wedge c_2)\vee x=1\}\\
            &=\{x\in A\mid (c_1\vee x)\wedge (c_2\vee x)=1\}\\
            &=\{x\in A\mid c_1\vee x=1 \mbox{ y }c_1\vee x=1\}\\
            &=\nabla(u_{c_1})\cap \nabla(u_{c_2})\\
            &=\alpha_A(c_1)\cap \alpha_A(c_2).
        \end{split}
        \]
        y
        \[
        \begin{split}
        \alpha_A(\bigvee X)&=\{x\in A\mid (\bigvee X)\vee x=1\}\\
        &=\bigvee\{x\in A\mid x'\vee x=1, x'\in X\}\\
        &=\bigvee \alpha_A[X],
        \end{split}
        \]
        donde $\{x\in A\mid x'\vee x=1, x'\in X\}$ dirigido.
        \end{enumerate}
        \item Por último, notemos que para un marco eficiente, el morfismo $\alpha_A$ es suprayectivo ya que si $F=\nabla(j)\in A^\wedge$, entonces
        \[
        v_F=u_d
        \]
        donde $d=v_F(0)\in A$, es decir, $\alpha_A(d)=\nabla(u_d)=\nabla(v_F)=\nabla(j)$
    \end{enumerate}
\end{proof}

\begin{cor}
    Consideremos $c\in \mathfrak{c}A$ tal que $u_c$ es ajustado, entonces $\alpha_A$ es inyectiva. 
\end{cor}

\begin{proof}
    Consideremos $\nabla(u_c)=\nabla(u_d)$, con $d\in \mathfrak{c}A$. De aquí que $\nabla(u_c)\subseteq \nabla(u_d)$ y $\nabla(u_c)\supseteq\nabla(u_d)$. Aplicando dos veces el Lema \ref{cn ajustado} tenemos que 
    \[
    u_c\leq u_d\quad\mbox{ y }\quad u_c\geq u_d.
    \]
    Por lo tanto, evaluando ambos núcleos en $0$ obtenemos $c=d$.
\end{proof}

Notemos que el morfismo $\alpha_A$ no siempre es suprayectivo, pero existen maneras de asegurar que esto ocurra.

\begin{lem}\label{Lema2.7}
    Para un marco eficiente y $c\in \mathfrak{c}A$ las siguientes son equivalentes
    \begin{enumerate}
        \item $\alpha_A$ es un isomorfismo.
        \item Si $u_c\in \mathfrak{C}A$, $u_c$ es ajustado.
        \item Para todo $c\in \mathfrak{c}A$ y $d, d'\in A_{u_c}$
        \begin{equation}\label{saju}
            d\leq d'\Leftrightarrow \nabla(u_d)\subseteq \nabla(u_{d'}).
        \end{equation}
    \end{enumerate}
\end{lem}

\begin{proof}
\begin{description}
        \item[$1)\Rightarrow 2)$] Si $c\in \mathfrak{c}A$, entonces $u_c\in NA$ y tomemos $F=\nabla(u_c)\in A^\wedge$. Sea $v_F$ el respectivo núcleo ajustado del bloque, es decir, $v_F\leq u_c$. Además, por la eficiencia $v_F=u_d$ y  $\nabla(u_d)=\nabla(u_c)$. Al ser $\alpha_A$ inyectiva, $d=c$. Por lo tanto $u_c$ es ajustado. 

        \item[$2)\Rightarrow 3)$] Supongamos que $\nabla(u_c)\in A^\wedge$ y que $u_c$ es ajustado para $c\in \mathfrak{c}A$. Notemos que la implicación $\Rightarrow)$ de (\ref{saju}) siempre es cierta, pues si $d\leq d'$ y $d\vee e=1$, entonces $d'\vee e=1$. Así solo basta probar la otra implicación.\\

        Queremos demostrar que $\nabla(u_d)\subseteq \nabla(u_{d'})$ implica que $d\leq d'$. Por el Lema \ref{cn ajustado}
        \[
        \nabla(u_d)\subseteq \nabla(u_{d'})\Leftrightarrow u_d\leq u_{d'}\Leftrightarrow d\leq d'.
        \]

        \item[$3)\Rightarrow 1)$] Si $A$ es un marco es eficiente, por el Lema \ref{Lema2.8y2.9}, se cumple que $\alpha_A$ es suprayectiva. Para la inyectividad, consideremos $\alpha_A(d)=\alpha_A(d')$, es decir, $\nabla(u_d)=\nabla(u_{d'})$. Luego, aplicando dos veces (\ref{saju}), se cumple que $d=d'$. Por lo tanto, $\alpha_A$ es inyectiva.
    \end{description}
\end{proof}

Los siguientes resultados son consecuencias del Lema \ref{Lema2.7}.

\begin{cor}
        Para un marco $KC$ las siguientes son equivalentes
    \begin{enumerate}
        \item $\alpha_A$ es un isomorfismo.
        \item Si $j\in \mathfrak{K}A$, $j$ es ajustado.
        \item Para todo $c\in \mathfrak{c}A$ y $d, d'\in A_{u_c}$
        \[
            d\leq d'\Leftrightarrow \nabla(u_d)\subseteq \nabla(u_{d'}).
        \]
    \end{enumerate}
\end{cor}

\begin{cor}
    Para todo $c\in \mathfrak{c}A$, $u_c$ es ajustado si y solo si $\alpha_A$ es isomorfismo y $A$ es eficiente.
\end{cor}

\begin{obs}
Si $A$ es un marco eficiente, $j\in \mathfrak{K}A$ y $F=\nabla(j)\in A^\wedge$, entonces $j\in [v_F, w_F]$. Por la eficiencia  
\[
u_d=v_F\leq j\leq w_F\Rightarrow \nabla(u_d)=\nabla(j),
\]
es decir, si consideremos $j\in \mathfrak{K}A$ le asignamos un elemento $d\in \mathfrak{c}A$ a través de
\[
\varphi\colon\mathfrak{K}A\to \mathfrak{C}A
\]
donde $\varphi(j)=u_d$.

\end{obs}

\textbf{PREGUNTA:} ¿Podemos caracterizar a los marcos eficientes por medio de las funciones $\alpha_A$ y $\varphi$?\\

Para $\mathfrak{Q}S=\{j\in \mathfrak{K}A\mid j \mbox{ es ajustado}\}$ tenemos 
\[\begin{tikzcd}
	{\mathfrak{K}A} && {\mathfrak{C}A} & {\mbox{ y }} & {\mathfrak{K}A} && {\mathfrak{Q}A}
	\arrow["\varphi", shift left=3, from=1-1, to=1-3]
	\arrow["i", shift left=3, from=1-3, to=1-1]
	\arrow["{\varphi'}", shift left=3, from=1-5, to=1-7]
	\arrow["\iota", shift left=3, from=1-7, to=1-5]
\end{tikzcd}\]
donde $\iota\colon \mathfrak{C}A\to \mathfrak{K}A$ es una inclusión.\\

Notemos que $\varphi$ y $\varphi'$ son morfismos que desinflan pues $\varphi(k),\varphi'(k)\leq k$. Además, son monótonos, ya que para $j, k\in \mathfrak{K}A$ con $j\leq k$, se cumple que $u_d=\varphi(j)\leq j\leq k$. De aquí que $\nabla(u_d)\subseteq\nabla(k)=\nabla(u_{d'})$, donde $u_{d'}=\varphi(k)$. Como $u_d$ es un núcleo ajustado
\[
\nabla(u_d)\subseteq \nabla(u_{d'})\Rightarrow u_d\leq u_{d'}.
\]
El mismo argumento se aplica para $\varphi'$.\\

Verifiquemos que $\varphi\dashv \iota$, es decir, para $j\in \mathfrak{K}A$ y $u_c\in \mathfrak{C}S$ se cumple que 
\[
\varphi(j)\leq u_c\Leftrightarrow j\leq \iota(u_c).
\]
La implicación $\Leftarrow)$ es trivial. Para la otra implicación, Supongamos que $\varphi(j)\leq u_c$ y $j\nleq \iota(u_c)=u_c$. Como $u_d$ es ajustado, es el menor elemento de su bloque y como $j\nleq u_c$. Tenemos dos opciones
\[
i)\;  u_d\leq u_c\leq j\leq w_F \quad \mbox{o}\quad ii)\;u_d\leq u_c\; \mbox{ y }\;u_c\notin [u_d, w_F].
\]
Si ocurre $i)$, entonces 
Además, si el marco $A$ es eficiente $\mathfrak{C}A\simeq \mathfrak{Q}A$. Entonces tenemos el diagrama 

\[
\begin{tikzcd}
\mathfrak{C}A \arrow[r, "i"] \arrow[rr, "\alpha", bend left] & \mathfrak{K}A \arrow[r, "\hat{\alpha}"] \arrow[l, "\varphi", bend left] & A^\wedge
\end{tikzcd}
\]
donde $\hat{\alpha}(j)=\nabla(j)$


\subsection{Algunos núcleos de $N\mathfrak{c}A$} 

Consideremos $A\in \Frm$ y $a, b\in A$. Sabemos que $a\prec b$ si y solo si existe $x\in A$ tal que $a\wedge x=0$ y $c\vee c=1$. En otras palabras, para $x=\neg a$, $x\vee b=1$.

\begin{dfn}\label{Bb en CA}
    Para $c\in \mathfrak{c}A$ y $a,b\in A_{u_c}$ decimos que $a\prec_c b$ para tener la relación \emph{bastante por debajo} en el marco $A_{u_c}$.
\end{dfn}

El elemento $(a\succ c)$ es la negación de $a$ en $A_{u_c}$. De aquí que
\[
a\prec_cb\Leftrightarrow b\vee (a\succ c)=1.
\]

Si $c\leq d$, entonces 
\[
A_{u_c}=\uparrow c\supseteq \uparrow d=A_{u_d}
\]
produce un morfismo de marcos $i_{dc}\colon A_{u_d}\to A_{u_c}$ dado por $i_{dc}(x)=d\vee x$. De esta manera para $c\leq d\leq e$ tenemos
\[\begin{tikzcd}
	{A_{u_e}} && {A_{u_d}} \\
	& {A_{u_c}}
	\arrow["{i_{ed}}"', from=1-3, to=1-1]
	\arrow["{i_{ec}}", from=2-2, to=1-1]
	\arrow["{i_{dc}}"', from=2-2, to=1-3]
\end{tikzcd}\]
y $i_{cc}\colon A_{u_c}\to A_{u_c}$ es la identidad.\\

La construcción anterior produce un funtor $F\colon \mathfrak{c}A\to \Frm$ donde 
\[
c\mapsto A_{u_c}\quad \mbox{ y }\quad c\leq d\mapsto i_{dc}
\]
son la asignación en objetos y flechas, respectivamente.\\

Consideremos un cono sobre $F$ dado por $(\hat{A}, \{\pi_c\colon \hat{A}\to A_{u_c}\}_{c\in \mathfrak{c}A})$. De manera similar que en la categoría de conjuntos, 
\[
\prod_{c\in \mathfrak{c}A}F(c)=\prod_{c\in \mathfrak{c}A}A_{u_c}=\{j\colon \mathfrak{c}A\to \bigcup A_{u_c}\mid j(c)\in A_{u_c}\}
\]
para todo $c\in \mathfrak{c}A$ y $j$ una función. De aquí que $j(c)\in A_{u_c}$ si y solo si $c\leq j(c)$, es decir, 
\[
j\in \prod_{c\in \mathfrak{c}A}A_{u_c} \Leftrightarrow c\leq j(c).
\]

Además, si $c\leq j(c)$ entonces $\pi_c(j)=j(c)=c\vee j(c)$, es decir, para $c\leq d$, $j(d)=i_{dc}(j(c))=d\vee j(c)$ y $j\colon \mathfrak{c}A\to \mathfrak{c}A$. Por lo tanto
\begin{equation}\label{compatibilidad}
\begin{split}
\hat{A}&=\{j\in \prod_{c\in \mathfrak{c}A}A_{u_c}\mid j(d)=i_{dc}(j(c))\}\\
&=\{j\colon \mathfrak{c}A\to \mathfrak{c}A\mid j(c)\in A_{u_c}\}
\end{split}
\end{equation}
para todo $c\leq d$. 

\begin{prop}\label{Flimite}
    Si $c\leq d\in \mathfrak{c}A$, entonces $\hat{A}$ es un cono límite sobre $F$.
\end{prop}

\begin{proof}
\begin{enumerate}
    \item Por construcción, cada proyección $\pi_c$ está bien definida. Además, si $c\leq d$
    \[
    i_{dc}\circ \pi_c(j)=i_{dc}(j(c))= d\vee j(c)=j(d)= \pi_d(j),
    \]
es decir, los triángulos del cono conmutan. Por lo tanto, $\hat{A}$ es un cono sobre $F$.

\item Sea $(Y,\{f_c\colon Y\to A_{u_c}\}_{c\in \mathfrak{c}A})$ cualquier otro cono sobre $F$, esto es,
para todo $c\le d$ se cumple $i_{dc}\circ f_c=f_d$. 
Consideremos el morfismo $u:Y\to \hat{A}$ definido por
\[
(u(y))(c)=f_c(y)\qquad (y\in Y,\ c\in \mathfrak{c}A).
\]
Primero vemos que $u(y)\in\lim F$ para cada $y\in Y$. Si $c\le d$, entonces
\[
(u(y))(d)=f_d(y)=(i_{dc}\circ f_c)(y)=i_{dc}(f_c(y))
=d\vee(u(y))(c),
\]
y por (\ref{compatibilidad}), $u(y)\in \hat{A}.$

Además, por definición de $u$ y de las proyecciones, para cada $c$ se tiene
\[
\pi_c\circ u \;=\; f_c,
\]
de modo que $u$ es un morfismo de conos $(Y\to \hat{A})$. Veamos que $u$ es único.\\

Si $v\colon Y\to \hat{A}$ es otro morfismo de conos con 
$\pi_c\circ v = f_c$ para todo $c$, entonces para todo $y\in Y$ y $c\in\mathfrak{c}A$,
\[
(v(y))(c)=\pi_c(v(y))=f_c(y)=(u(y))(c).
\]
De aquí que $v(y)=u(y)$ para todo $y$, luego $v=u$.
\end{enumerate}
Por lo tanto $\hat{A}$ satisface la propiedad universal del cono límite.
\end{proof}

\begin{lem}\label{Lema3.1}
    Para cualquier $A\in \Frm$ y cada función $j\colon \mathfrak{c}A\to \mathfrak{c}A$ las siguientes son equivalentes:
    \begin{enumerate}
        \item $j\in \hat{A}$.
        \item $j(d)=j(c)\vee d$ para todo $d\geq c\in \mathfrak{c}A$.
        \item $j(c\vee a)=j(c)\vee a$ para todo $c\in \mathfrak{c}A$ y $a\in A$.
        \item $j(c\vee e)=j(c)\vee e$ para todo $c, e\in \mathfrak{c}A$.
    \end{enumerate}
\end{lem}

\begin{proof}
    \begin{description}
        \item[$1)\Leftrightarrow 2)$] Se cumple por la construcción de $\hat{A}$.
        \item[$2)\Rightarrow 3)$] Tomando $c\leq d=c\vee a$, $d\in \mathfrak{c}A$, pues $\mathfrak{c}A$ es una sección superior. De esta manera, por $2)$
        \[
        j(d)=j(c)\vee d=j(c)\vee c\vee a=j(c)\vee a,
        \]
        pues $c\leq j(c)$.
        \item[$3)\Rightarrow 4)$] Si $e\in \mathfrak{c}A$, $e\in A$ y aplicando $3)$ obtenemos lo que queremos.
        \item[$4)\Rightarrow 2)$] Si $c\leq d$, se cumple que $c\vee d=d$, tomando $e=d$ y aplicando $4)$ se tiene $j(d)=j(c)\vee d$. 
    \end{description}
\end{proof}

\begin{lem}\label{Lema3.3}
    Para $A\in \Frm$ si $j\in \hat{A}$, entonces $j$ es un núcleo en $\mathfrak{c}A$.
\end{lem}

\begin{proof}
    Verifiquemos que $j\in \hat{A}$ cumple las condiciones de núcleo.
    \begin{enumerate}
        \item Por construcción, $c\leq j(c)$, es decir, $j$ infla.
        \item Notemos que para $d=j(c)$, entonces $j(d)=j(j(c))=j(c)\vee j(c)=j(c)$. Por lo tanto $j$ es idempotente.
        \item Consideremos $c=d\wedge d'$, entonces $c\leq d, d'$. Luego
        \[
        j(d)=j(d\wedge d')\vee d\quad \mbox{ y }\quad j(d')=j(d\wedge d')\vee d'.
        \]
        De aquí que 
        \[
        \begin{split}
            j(d)\wedge j(d')& =(j(d\wedge d')\vee d)\wedge (j(d\wedge d')\vee d')\\
            & =j(d\wedge d')\vee (d\wedge d')\\
            &=j(d\wedge d')
        \end{split}
        \]
        donde la segunda igual se da por la distributividad y la última del hecho de que $j$ infla. Así $j$ respeta ínfimos.
    \end{enumerate}
    Por lo tanto, $j$ es un núcleo en $\mathfrak{c}A$.
\end{proof}

Sabemos que si $j\in NA$, entonces $j^{-1}(1)$ es un filtro.

\begin{lem}\label{Lema3.3}
    Para cualquier núcleo $j$ en una semiretícula de Heyting $A$ y cualquier $a\in A$, la desigualdad $v_a\leq j$ se cumple si y solo si $j(a)=1$, donde $v_a(x)=(a\succ x)$.
\end{lem}

\begin{proof}
Es consecuencia de las propiedades de los $v$-núcleos.
\end{proof}

Enunciamos un lema auxiliar para la prueba del siguiente teorema.

\begin{lem}\label{lem:nucleo-dirigidos-a-arbitrarios}
Sea $A$ un marco y $j\colon A\to A$ un núcleo. 
Si $j$ preserva supremos finitos y supremos dirigidos, entonces $j$ preserva todos los supremos arbitrarios, es decir,
\[
j(\bigvee X)=\bigvee j[X]
\]
para todo $X\subseteq A$.
\end{lem}

\begin{proof}
Sea $X\subseteq A$ y consideremos el conjunto
\[
\mathcal{F}(X)=\{\bigvee F\mid F\subseteq X\text{ es finito}\}.
\]
Notemos que $\mathcal{F}(X)$ es dirigido pues $u=\bigvee F$ y $v=\bigvee G$ con $F,G$ finitos, 
entonces $w=\bigvee(F\cup G)\in\mathcal{F}(X)$ y $u\le w$, $v\le w$.

Además, 
\[
\bigvee X\;=\;\bigvee \mathcal{F}(X).
\]

Por hipótesis $j$ preserva supremos finitos y dirigidos, entonces
\[
j(\bigvee X)=j(\bigvee \mathcal{F}(X))=\bigvee_{u\in \mathcal{F}(X)} j(u)=\bigvee_{F\subseteq_{\mathrm{fin}} X} j(\bigvee F)=\bigvee_{F\subseteq_{\mathrm{fin}} X} \bigvee j[F].
\]
Finalmente, como $\{\bigvee j[F]\mid F\subseteq_{\mathrm{fin}} X\}$ es dirigido y su supremo coincide con $\bigvee j[X]$, tenemos que
\[
j\!\Big(\bigvee X\Big)\;=\;\bigvee j[X].
\]
\end{proof}

\begin{thm}\label{Teorema3.4}
    Para $A$ un marco Hausdorff y cualquier núcleo $j\colon \mathfrak{c}A\to \mathfrak{c}A$ las siguientes son equivalentes
    \begin{enumerate}
        \item $j\in \hat{A}$.
        \item $j$ preserva supremos no vacíos.
        \item $j$ es Scott-continuo.
        \item $\nabla(j)\in \mathfrak{c}A^\wedge$.
    \end{enumerate}
\end{thm}

\begin{proof}
    \begin{description}
        \item[$1)\Rightarrow 2)$] Consideremos $c, d\in \mathfrak{c}A$, entonces
        \[
        j(c\vee d)=j(c)\vee c\vee d=j(c)\vee d\leq j(c)\vee j(d).
        \]
        Por monotonía se cumple que $j(c)\vee j(d)\leq j(c\vee d)$. Por lo tanto $j(c\vee d)=j(c)\vee j(d).$

        Para los supremos dirigidos, consideremos $\mathcal{D}\subseteq \mathfrak{c}A$ dirigido y tomemos $c\in \mathcal{D}$. Como $\bigvee \mathcal{D}\in \mathfrak{c}A$ se cumple que 
        \[
        j(\bigvee \mathcal{D})=j(c\vee \bigvee \mathcal{D})=j(c)\vee\bigvee \mathcal{D}=\bigvee_{d\in \mathcal{D}}j(c)\vee d=\bigvee_{d\in \mathcal{D}}j(c\vee d)=\bigvee_{e\in \mathcal{D}}j(e)
        \]
        donde la última igual se cumple por ser $\mathcal{D}$ dirigido. 

        De aquí que, por el Lema \ref{lem:nucleo-dirigidos-a-arbitrarios} se cumple que $j$ preserva cualquier supremo.

        \item[$2)\Rightarrow 3)$] Si $j$ preserva supremos dirigidos, $j$ es Scott continuo.

        \item[$3)\Rightarrow 4)$] Por la prueba del Lema \ref{Lema2.3} $5)$, como $\mathfrak{c}A$ es compacto se tiene que para $X\subseteq \mathfrak{c}A$, con $X$ dirigido, existe $x\in X$ tal que $x=1$ y $x\in A_{u_c}$ para $c\in \mathfrak{c}A$ y $1=\bigvee X$. como $j$ es Scott continuo
        \[
        j(\bigvee X)=\bigvee j[X]=1,
        \]
        es decir, $\bigvee X\in \nabla(j)$. de aquí que existe $j(x)\in j[X]$ tal que $j(x)=1$, entonces $x\in \nabla(j)$. Por lo tanto, $X\cap\nabla(j)\neq\emptyset$ y $\nabla(j)\in \mathfrak{c}A^\wedge$.

        \item[$4)\Rightarrow 1)$] Sabemos que $j\in \lim F$ si $j(d)=j(c)\vee d$ para $c\leq d$. Como $j$ es monótona
        \[
        j(c)\leq j(d)\Rightarrow d\vee j(c)\leq d\vee j(d)=j(d)
        \]
        siempre se cumple. De esta manera, basta con verificar que $j(d)\leq d\vee j(c)$. Consideremos $e\in A_{u_c}$ tal que $e\vee j(d)=1$ y como $e\vee j(d)\leq j(e\vee d)$, entonces $j(e\vee d)=1$, es decir, $e\vee d\in \nabla(j)$.
    \end{description}
\end{proof}

\section{Cosas de Ángel}
Trivially {\rm KC} implies patch trivial (or equivalently tidy) we want some converse of this fact.



Following articulo de igor.,

\begin{dfn}\label{pfit}
A frame $A$ has \emph{fitted points} ({\rm p-fit} for short) if for every point $p\in\pt(A)$ the nucleus \[{\rm w}_{p} \text{ is fitted }\]
that is, to said for every point $p$ the nucleus ${\rm w}_{p}$ is alone in its block.
\end{dfn}


%QUIERO UN TEOREMA TIPO FIT Y SUBFIT PARA p-fit frames%

In general for each $p\in\pt(A)$, the nucleus ${\rm w}_{p}$ is the largest memeber of his block, that is,
\[[v_{\mathcal{P}},{\rm w}_{p}]\] the corresponding block, here $\mathcal{P}=\{x\in A\mid x\nleq p\}$ 
in this case we know how to calculate \[v_{\mathcal{P}}.\] using the prenucleus $f_{\mathcal{P}}$
we know that \[v_{\mathcal{P}}=f_{\mathcal{P}}^{\infty}=(\dot{\bigvee}\{{\rm v}_{x}\mid x\in\mathcal{P}\})^{\infty}\]
moreover:

\[
f_{\mathcal{P}}(x)=\left\{
	\begin{array}{lcc}
1 & \mbox{ si } & x\nleq p\\
\\
\leq p & \mbox{ si } & x\leq p
\end{array}\right.
\]
for $x\in A$.


and in fact ${\rm w_{p}}=u_{p}\vee v_{\mathcal{P}}=f_{\mathcal{P}}\circ u_{p}$. If ${\rm w_{p}}$ is fitted, that is, \[{\rm w_{p}}=v_{\mathcal{P}}\]

then one need to have $u_{p}\leq v_{\mathcal{P}}$ then \[p\leq v_{\mathcal{P}}(0)\] by the equation of $f_{\mathcal{P}}$ we have \[0\leq\cdots\leq f_{\mathcal{P}}^{\alpha}(0)\leq\cdots\leq\]


\begin{prop}\label{pfitequiv}
Let $A$ be a frame for each $p\in\pt(A)$ the following are equivalent:
 \begin{itemize}
\item[(i)] ${\rm w}_{p}$ is fitted.
\item[(ii)] ${\rm w}_{p}$ is alone in its block.
\item[(iii)] $u_{p}\leq v_{\mathcal{P}}$.
\item[(iv)] $u_{p}\leq f_{\mathcal{P}}$.
\item[(v)] $f_{\mathcal{P}}\circ u_{p}=v_{\mathcal{P}}$.
\item[(vi)] aqui debe de ir una formula de primer de orden.
 \end{itemize}
\end{prop}	


\begin{prop}\label{fitpro}
In a {\rm p-fit} frame for each $p\in\pt(A)$ the nucleus ${\rm w}_{p}$ is a maximal element in $\mathrm{p}A$.
\end{prop}



\begin{proof}\label{maxp}
First we dealing with the basics $v_{F}$ for $F\in A^{\wedge}$ of the patch frame, given any ${\rm w}_{p}$ suppose that  
${\rm w}_{p} \leq v_{F}$ then by (propiedades generales de los w) $v_{F}={\rm w}_{b}$ where $b=v_{F}(0)$ thus \[{\rm w}_{p}\leq {\rm w}_{b}\Leftrightarrow {\rm w}_{p}(b)=b\]
 since ${\rm w}_{p}$ is two valuated we have $b=1$ or $b=p$  if the first case occur then we are done, for the case $b=p$ we have $v_{f}(p)=p$ that is, to say, $p\notin F$,
 then by the Birkhoff's separation lemma we can find a completely prime filter $D$ such that \[F\subseteq G\niton p\] let $q$ the corresponding point associated to $G$, then $p\leq q$ since $A$ is 
 {\rm p-fit} $v_{G}={\rm w}_{q}$ and thus ${\rm w}_{p}\leq {\rm w}_{q}$ wich is equivalent to ${\rm w}_{p}(q)=q$ again since we are dealing with points one neccesary has $p=q$.

 Now consider any closed ${\rm u}_{c}$ such that, ${\rm w}_{p}\leq {\rm u}_{c}$ then ${\rm w}_{p}(c)=1$ and thus $1=c$.

 Therefore in basics of the patch the nuclei ${\rm w}_{p}$ are maximal, now consider any $k\in\mathrm{p}A$ such that $k\in\mathfrak{K}A$
\end{proof}


\begin{prop}\label{kcf}
Let $A$ be a frame then if \[v_{F}\neq v_{G}\]
\end{prop}




\begin{dfn}\label{tame}
A frame $A$ is \emph{tame} if does not have wild points.
\end{dfn}


\begin{prop}\label{fitpro1}
	In a tame {\rm p-fit} frame the patch frame  $\mathrm{p}A$ is ${\rm T}_{1}$.
	\end{prop}

Since every hausdorff frame is tame and {\rm p-fit} we have:

	\begin{cor}\label{fitpro2}
		If $A\in\EuScript{H}rm$ then, the patch frame  $\mathrm{p}A$ is ${\rm T}_{1}$.
	\end{cor}	





\begin{dfn}\label{KQ}
Let $A$ be a frame a nucleus $k$ on $A$ it said to be \emph{kq} if $A_{j}$ is a compact frame. 
\end{dfn}
Denote by 
\[
\mathfrak{K}A=\{j\in NA\mid j \text{ is } kq\}.
\]

\begin{dfn}\label{KCH}
A frame $A$ is \emph{compact closed Hausdorff} ({\rm KCH} for short) if every compact quotient of $A$ is closed and Hausdorff.
\end{dfn}



Denote by $\mathfrak{f}A=\{kq\text{ fitted nuclei }\}=\{v_{F}\mid F\in A^{\wedge}\}$




denote by $\mathfrak{C}A=\{a\in A\mid {\rm u}_{a}\in\mathfrak{K}A\}$

\section{Resultados adicionales}

\begin{lem}
    Sea $p\in A$. $p$ es $\wedge-$irreducible si y solo si $A$ está linealmente ordenado.
\end{lem}

\begin{proof}
\begin{description}
    \item[$\Rightarrow )$] Consideremos $p, q$ elementos $\wedge-$irreducibles. Veamos que $p\leq q$ o $q\leq p$. Notemos que 
    \[
    p\wedge q\leq p\qquad \mbox{ y }\qquad p\wedge q\leq q 
    \]
    Por hipótesis, $p$ y $q$ son $\wedge-$irreducibles, es decir, 
    \[
    p\leq p \mbox{ o }q\leq p\quad\mbox{ y }\quad p\leq q \mbox{ o }q\leq q.
    \]
    Por lo tanto $p\leq q$ o $q\leq p$, es decir, $A$ está linealmente ordenado.

    \item[$\Leftarrow )$] Consideremos $a,b, p\in A$ tales que $a\wedge b\leq p$. Cómo $A$ es linealmente ordenado, se cumple que $a\wedge b=b$ o $a\wedge b=a$ (pues $a\leq b$ o $b\leq a$). Por lo tanto $a\leq p$ o $b\leq p$. Por lo tanto, $p$ es $\wedge-$irreducible. Como $p$ es arbitrario, entonces ocurre para cualquier $p\in A$.
\end{description}
\end{proof}

\begin{obs}
    Los marcos linealmente ordenados no son Hausdorff punteados. 
\end{obs}

\begin{proof}
    En los marcos linealmente ordenados todos los elementos son semiprimos. Por lo tanto existen primos que no son máximos.
\end{proof}

The block structure on a frame is an important problem and its related with some separation properties of frames.
%ESTO SE USA ???????????????%
\begin{prop}\label{morfismo}
For $F\in A^\wedge$ and $Q\in\mathcal{Q}S$, if $j\in [v_Q, w_Q]$, then $U_*jU^*\in [v_F, w_F]$, where $U^*$ is the morfism spatial reflection $U_*$ is the right adjoint.
\end{prop}

\begin{proof}
Since $N$ is a functor, we have 
\[\begin{tikzcd}
	A & NA \\
	\\
	{\mathcal{O}S} & {N\mathcal{O}S}
	\arrow[""{name=0, anchor=center, inner sep=0}, "U"', from=1-1, to=3-1]
	\arrow[""{name=1, anchor=center, inner sep=0}, "{N(U)}", from=1-2, to=3-2]
	\arrow["{N(\_)}", shorten <=7pt, shorten >=7pt, maps to, from=0, to=1]
\end{tikzcd}\]
and $N(U)_*$ is the right adjoint of $N(U)^\wedge$. Note the following:
\begin{enumerate}
	\item $N(U)(j)\leq k\Leftrightarrow j\leq N(U)_*k$.
	\item If $k\in N\mathcal{O}S$ then $N(U)(j)\leq k\Leftrightarrow Uj\leq kU$.
	\item $N(U)_*k=U_*kU^*$ and $UN(U)_*k=k(U)$.
\end{enumerate}
In 3), if $j=k$, $N(U)_*(j)=U_*jU^*$ and $UN(U)_*j=jU$. For $x\in F$
\[\begin{tikzcd}
	x\in A & {\mathcal{O}S} & {\mathcal{O}S} & A
	\arrow["{U^*}", from=1-1, to=1-2]
	\arrow["j", from=1-2, to=1-3]
	\arrow["{U_*}", from=1-3, to=1-4]
\end{tikzcd}\]
and $U_*(j(U(x))=\bigwedge(S\setminus j(U(x)))$. Note that $U_*(j(U^*(x)))\subseteq \pt A$. Thus

\[
\begin{split}
x\in F \Leftrightarrow Q\subseteq U(x) &\Leftrightarrow U(x)\in \nabla(j)=\nabla(Q)\Leftrightarrow S\setminus j(U(x))=\emptyset\\
& \Leftrightarrow \bigwedge (S\setminus j(U(x)))=1=(U_*jU^*)(x)\\
&\Leftrightarrow x\in \nabla(U_*jU^*)
\end{split}
\]
Therefor $F=\nabla(U_*jU^*)$.
\end{proof}
In this way we have a function 
\[
\mho\colon [V_Q, W_Q]\to [V_F, W_F]
\]


%\begin{prop}\label{Bloqtri}
%For every $A\in \EuScript{Hrm}rm$ the interval corresponding to the block determined by a open filter $F\in A^{\wedge}$ is trivial, that is,\[[v_{F},w_{F}]=\{*\}\]

%\end{prop}

%\begin{proof}
 % We know that for all $F\in A^\wedge$ the following holds: $v_F\leq w_F$. As a contradition, suppose that exists $F\in A^\wedge$ such that $w_F\nleq v_F$, that is, exists $a\in A$ such that $w_F(a)\nleq v_F(a)$.\\

  %Note that $w_F(a)\neq 1$, otherwise 
  %\[
  %1=w_F(a)=\bigwedge \{p\in M\mid a\leq p\}\leq p
  %\]
  %and this is a contradition because $p\neq 1$.\\

%Then $1\neq w_F(a)\nleq v_F(a)$ and for the property ($\mathbf{H}$), exists $u\in A$ such that
%\begin{equation}\label{vFwF en a}
%u\nleq w_F(a)\quad \mbox{ and }\quad \neg u \nleq v_F(a)
%\end{equation}

 %Due to monotony, $w_F(0)\leq w_F(a)$ and $v_F(0)\leq v_F(a)$- Thus, for (\ref{vFwF en a}) we have that
%\begin{equation}\label{vFwF en 0}
%i)\,u\nleq w_F(0)\quad \mbox{ and }\quad ii)\,\neg u\nleq v_F(0).
%\end{equation}

%For (\ref{vFwF en 0})-$(i)$ is true that $u\nleq \bigwedge M$, in particular, $u\nleq p$ for some $p\in M$. Therefore, $\neg u\leq p$ and $\neg u\leq w_F(0)$. If (\ref{vFwF en 0})-$(ii)$ is true, then $u\notin F$, in otherwise 
%\[
%u\in F\Rightarrow v_u\leq f \Rightarrow v_u(0)=\neg u\leq f(0)
%\] 
%and this is a contradition. Thus, for the Birkhoff's separation Lemma, exists a completely prime filter $G$ such that $u\notin G\supseteq F$. We take
%\[
%q=\bigvee \{y\in A\mid y\notin G\}
%\]
%the point corresponding to $G$. Thus, $u\notin G$, $u\leq q$. If $q\notin F$, then $q\in M$ and $u\nleq q$. Hence $u\leq q$, $u\nleq q$ and this is a contradition.
%\end{proof}

%A consequence of the Proposition \ref{Bloqtri} is that $v_F=w_F$, so that, $A_{v_F}=A_{w_F}$ and $A_{w_F}\simeq \mathcal{O}M\simeq \mathcal{O}Q$. Thus, for all $j\in KA$ we have that $j=v_F$. Then in the Huasdorff case
%For \ref{vFwF en 0}-$(i)$ is true that $u\nleq \bigwedge M$, in particular, $u\nleq p$ for all $p\in M$. Therefore, $\neg u\leq p$ and $\neg u\leq w_F(0)$. If \ref{vFwF en 0}-$(ii)$ is true, then $u\notin F$, in otherwise 
%\[
%u\in F\Rightarrow v_u\leq f \Rightarrow v_u(0)=\neg u\leq f(0)
%\] 
%and this is a contradition. Thus, for the Birkhoff's separation lemma, exists a completely prime filter $G$ such that $u\notin G\supseteq F$. We take
%\[
%q=\bigvee \{y\in A\mid y\notin G\}
%\]
%the point corresponding to $G$. Thus, $u\notin G$, $u\leq q$. If $q\notin F$, then $q\in M$ and $u\nleq q$. Hence $u\leq q$, $u\nleq q$ and this is a contradition.
%\end{proof}

%\[\begin{tikzcd}
%	A & {A_F} \\
%	{\mathcal{O}S} & {\mathcal{O}S_\nabla}
%	\arrow[from=1-1, to=1-2]
%	\arrow[from=1-1, to=2-1]
%	\arrow["g", from=1-2, to=2-2]
%	\arrow[from=2-1, to=2-2]
%\end{tikzcd}
%\]
%where $g$ (seria el h del \ref{bigvee g})is an isomorphism and $A_F\simeq \mathcal{O}Q\simeq \mathcal{O}S_\nabla$.\\
%HAY que explicar mejor los isorfismos quien es M quien es 
%Q
%On the other hand, $U_*u_{Q'}U^*=v_F$ if and only if $u_{Q'}U^*=U^*v_F$, for the adjuntion properties and $U^*$ the spatial reflection morphism. Therefore

%\[\begin{tikzcd}
%	A & A \\
%	{\mathcal{O}S} & {\mathcal{O}S}
%	\arrow["{v_F}", from=1-1, to=1-2]
%	\arrow["U", shift left=2, from=1-1, to=2-1]
%	\arrow["U"', shift right=2, from=1-2, to=2-2]
%	\arrow["{U_*}"{pos=0.6}, shift left=3, from=2-1, to=1-1]
%	\arrow["{v_\nabla}"', from=2-1, to=2-2]
%	\arrow["{U_*}"', shift right=2, from=2-2, to=1-2]
%\end{tikzcd}\]
%so that, if $A\in \EuScript{Hrm}rm$, then $KC$ implies patch trivial.\\

%The above is the proof of the following theorem.

%\begin{thm}\label{C.Hausdorff}
%If $A\in \EuScript{Hrm}rm$. then every compact quotient is isomrphic to a closed quotient of the topology of a Hausdorff space.
%\end{thm}




%\begin{cor}\label{Viet}
%	If $A\in \EuScript{Hrm}rm$.
%	\[\EuScript{Q}(S)\cong\pt(V(A))\] 
%	\end{cor}

%\begin{prop}\label{Himplies pt}
%Every Hausdorff frame $A$ (in the sense of Johnstone and Shou) is tidy, that is, $A$ is patch trivial. 
%\end{prop}

\begin{thm}\label{haus1}
Let $A\in\EuScript{H}rm$ then for every $F\in A^{\wedge}$ with corresponding $\mathcal{Q}$ compact saturated we have \[\mathcal{O}\mathcal{Q}\cong\uparrow\mathcal{Q}'\], that is,
the frame of opens of the point space of $A_{F}$ is isomorphic to a compact closed quotient of a Hausdorff space.
\end{thm}
%AUN NO ME GUSTA COMO ESTA REDACTADO ESTO%


\begin{proof}



\end{proof}

EJEMPLOS DE marcos pt que no sean KC

HAY que COMENTAR LAS COSAS QUE ESTAN MAL comentar me refiero a ponerlas entre 

%%
	
\cite{escardo2001regular} \cite{escardo2006compactly}


\cite{sexton2006point}


\bibliographystyle{amsalpha}

\bibliography{research2}


\end{document}