%\documentclass[1p]{elsarticle}
\documentclass[11pt]{amsart}
\usepackage[mathscr]{eucal}
\usepackage{times}
\usepackage{amsmath,amssymb,latexsym,amscd}   
\usepackage{hyperref}
\hypersetup{colorlinks=true,linkcolor=blue,citecolor=red,linktocpage=true}
%\input xypic
%\xyoption{all}
\usepackage{graphicx}
\usepackage[all,cmtip]{xy}
\usepackage{fancyhdr}
\usepackage{mathalfa}
\usepackage{mathrsfs}
%\usepackage[sans]{dsfont}
\usepackage{upgreek}
%\usepackage{mathpazo}
\usepackage{tikz}
\usepackage{graphicx}
\usepackage{tikz-cd}
%%%%%%%%%%%%%%%%%%

\DeclareMathOperator{\op}{op}
\DeclareMathOperator{\spec}{spec}
\DeclareMathOperator{\pt}{pt}
\DeclareMathOperator{\Sp}{sp}
\DeclareMathOperator{\Frm}{Frm}


\theoremstyle{plain}
\newtheorem*{theorem*}{Theorem}
\newtheorem{thm}{Theorem}[section]
\newtheorem{cor}[thm]{Corollary}
\newtheorem{lem}[thm]{Lemma}
\newtheorem{prop}[thm]{Proposition}

\theoremstyle{definition}
\newtheorem{dfn}[thm]{Definition}
\newtheorem{obs}[thm]{Remark}
\newtheorem{ej}[thm]{Example}
  %%
    %\providecommand{\corollaryname}{Corollary}
  %\providecommand{\examplename}{Example}
  %\providecommand{\lemmaname}{Lemma}
  %\providecommand{\propositionname}{Proposition}
  %\providecommand{\remarkname}{Remark}
%\providecommand{\theoremname}{Theorem}
%\providecommand{\subexamplename}{Subexample}
  %\providecommand{\definitionname}{Definition}
\begin{document}
%\begin{frontmatter}

\title{The patch frame and its relations with separation in point-free topology}


\author{}
\email{}
\address{}


\address{}
%$\cortext[cor1]{Corresponding author}

\begin{abstract}

\end{abstract}

\maketitle
\section{Introduction}
Aquí va la introducción.
\section{Preliminaries}\label{pre}





\section{Hausdorff properties implies patch triviality}
Gadgets:

$A$ the base frame

Its point space $S=\pt(A)$.

$NA$ is the assembly of nucleus of $A$.
 
The compact saturated sets of $S$, \[\mathcal{Q}(S).\]

The preframe of open filters of $A$,
\[A^{\wedge}.\]

The preframe of open filters of $\Omega(S)$.

\[\Omega(S)^{\wedge}.\]

The set of compact quotients
\[
KA=\{j\in NA\mid A_j \mbox{ is compact}\}.
\]

First we recall that every open filter $F\in A^{\wedge}$ has three faces, that is, determines (and its determine) by :

\begin{itemize}
\item The compact saturated $Q\in \mathcal{Q}S$.
\item $\nabla\in\Omega(S)^{\wedge}$.
\item The compact quotient $A\rightarrow A_{F}$.
\item The fitted nucleus $v_{F}$.
\end{itemize}
Hoffman-Mislove can be rephrase:

\[\text{There is a bijection between compact quotients of } A \text{ and compact saturated sets of }S\]

\section{compact quotients}


\begin{dfn}\label{KOMPACT}
	A frame has $\mathrm{KC}$ if every compact quotient of $A$ is a closed one. In other words every compact sublocale is closed.
\end{dfn}
	
Denote by $\EuScript{Hrm}rm$ the subcategory of $Frm$ of Hausdorff frames in the sense of Johnstone and Shu.\\

If $f^*\colon A\to B$ is a frame morphism and $F\subseteq A$, $G\subseteq B$ filters in $A$, $B$, respectively, we can produce new filters as follows
\begin{equation}\label{Imagenfiltros}
b\in f^*F \Leftrightarrow f_*(b)\in F\quad \mbox{ and}\quad a\in f_*G \Leftrightarrow f^*(a)\in G
\end{equation}
where $a\in A, b\in B$ and $f_*$ is the right adjoint of $f^*$. Here $f^*F\subseteq B$ and $f_*G\subseteq A$ are filters on $B$ and $A$, respectively.\\

\begin{prop}\label{fF}
For $f=f^*\colon A\to B$ a frame morphism and $G\in B^\wedge$, then $f_*G\in A^\wedge$.
\end{prop}

\begin{proof}
By (\ref{Imagenfiltros}), $f_*G$ is a filter on $A$. We need $f_*G$ to satisfy the open filter condition. Let $X\subseteq A$ be such that $\bigvee X\in f_*G$, with $X$ directed. Then
\[
Y=\{f(x)\mid x\in X\}
\] 
is directed and $f(\bigvee X)=\bigvee f[X]=\bigvee Y\in G$. Since $G$ is a open filter, exists $y=f(x)\in Y$ such that $y\in G$. Thus $x\in f_*G$, so that, $f_*G\in A^\wedge$.
\end{proof}

In \cite{sexton2003point}, the autor says that $A\in \mathbf{Frm}$ is \emph{tidy} if for all $F\in A^\wedge$
\[
x\in F\Rightarrow u_d(x)=d\vee x=1
\]
where $d=d(\alpha)=f^\alpha(0)$, $f=\dot{\bigvee}\{v_y\mid y\in F\}$, $v_y\in NA$ and $0=0_A$ (the reason for the last two clarifications will be understood later).\\

We want translate this same notion, but for $A_j$ when $j\in NA$, so that, for all $F\in A_j^\wedge$, if $x\in F$ then $d\vee x=1$, with $d$ similar to before, because for this case we have that $v_y\in NA_j$ and $0_{A_j}=j(0)$.\\

Simmons proves in \cite{simmons2004vietoris} (Lamma 8.9 and Corollary 8.10), that the diagram
\[\begin{tikzcd}
	A & A \\
	{\mathcal{O}S} & {\mathcal{O}S}
	\arrow["{f^\infty}", from=1-1, to=1-2]
	\arrow["{U_A}"', from=1-1, to=2-1]
	\arrow["{U_A}", from=1-2, to=2-2]
	\arrow["{F^\infty}"', from=2-1, to=2-2]
\end{tikzcd}\]
commutes laxly, so that, $U_A\circ f^\infty\leq F^\infty \circ U_A$. In this diagram $U_A$ is the spatial reflection morphism, $f^\infty$ and $F^\infty$ represent the associated nuclei to the filters $F\in A^\wedge$ and $\nabla\in \mathcal{O}S^\wedge$. Also $f^\infty$ and $F^\infty$ are idempotent closeds associated to the prenuclei $f$ and $F$ respectively.\\

We prove something more general here, since we consider the diagram
\[\begin{tikzcd}
	A & A \\
	{A_j} & {A_j}
	\arrow["{\hat{f}^\infty}", from=1-1, to=1-2]
	\arrow["j"', from=1-1, to=2-1]
	\arrow["j", from=1-2, to=2-2]
	\arrow["{f^\infty}"', from=2-1, to=2-2]
\end{tikzcd}\]
where $\hat{f}^\infty$ is the nuclei associated to the filter $j_*F\in A^\wedge$ and $j\in NA$.  

\begin{lem}\label{f1f}
    For $j$, $f$ and $\hat{f}$ as above, it holds that $j\circ \hat{f}\leq f\circ j$.
\end{lem}
\begin{proof}
    By (\ref{Imagenfiltros}) is true that
    \[
    \hat{f}=\dot{\bigvee}\{v_y\mid y\in j_*F\}\quad  \mbox{ and } \quad f=\dot{\bigvee}\{v_{j(y)}\mid j(y)\in F\}. 
    \]
then, for $a\in A$ it is hold
\[
v_y(a)=(y\succ a)\leq \hat{f}(a)\leq j(\hat{f}(a)).
\]

Also, for all $a, y\in A$, $(y\succ a)\wedge y=y\wedge a$ and
\[
\begin{split}
j((y\succ a)\wedge y)\leq j(a) & \Leftrightarrow j(y\succ a)\wedge j(y)\leq j(a)\\
& \Leftrightarrow j(y\succ a)\leq (j(y)\succ j(a)).
\end{split}
\]

Thus 
\[
v_y(a)\leq j(\hat{f}(a))\leq (j(y)\succ j(a))=v_{j(y)}(j(a))\leq f(j(a)).
\]

Therefore $j\circ \hat{f}\leq f\circ j$.
\end{proof}

Now, we prove the above, but for all $\alpha$-ordinals.\\

\begin{cor}\label{finftyf}
    For $j$, $f$ and $\hat{f}$ as before, it is hold that $j\circ \hat{f}^\alpha\leq f^\alpha\circ j$
\end{cor}

\begin{proof}
    For an ordinal $\alpha$ we will check that $j\circ \hat{f}^\alpha\leq f^\alpha\circ j$. We will do it by transfinite induction.\\

    If $\alpha=0$, it is trivial.\\

    For the induction step, we assume that for $\alpha$ it holds. Then
    \[
    j\circ \hat{f}^{\alpha+1}=j\circ \hat{f}\circ \hat{f}^{\alpha}\leq  f\circ j\circ \hat{f}^\alpha\leq f\circ f^\alpha\circ j=f^{\alpha+1}\circ j,
    \]
    where the first inequality is Lemma \ref{f1f} and the second is true by the induction hypothesis.\\

    If $\lambda$ is a limit ordinal, then
	\[
	\hat{f}^\lambda=\bigvee\{\hat{f}^\alpha\mid \alpha<\lambda\}, \quad f^\lambda=\bigvee\{f^\alpha\mid \alpha<\lambda\} 
	\]
	and 
	\[
		j\circ \hat{f}^\lambda=j\circ \bigvee_{\alpha<\lambda}\hat{f}^\alpha\leq\bigvee_{\alpha<\lambda}j\circ \hat{f}^\alpha.
	\]
	Thus, by the induction hypothesis, we have that
	\[
	j\circ \hat{f}^\alpha\leq f^\alpha\circ j\Rightarrow \bigvee_{\alpha<\lambda}j\circ \hat{f}^\alpha\leq \bigvee_{\alpha<\lambda} f^\alpha\circ j.
	\]
	Therefore $j\circ \hat{f}^\lambda\leq f^\lambda\circ j$.
\end{proof}

By the Corollary \ref{finftyf}, we have that $j\circ \hat{f}^\infty\leq f^\infty\circ j$ is true. Futhermore, by H-M Theorem, $f^\infty=v_F$ and $\hat{f}^\infty=v_{j_*F}$. With this in mind, we have the following diagram
\[\begin{tikzcd}
	A && {A_{j_*F}} \\
	\\
	{A_j} && {A_F}
	\arrow["{(v_{j_*F})^*}", shift left, from=1-1, to=1-3]
	\arrow["j"', from=1-1, to=3-1]
	\arrow["H"', from=1-1, to=3-3]
	\arrow["{(v_{j_*F})_*}", shift left, from=1-3, to=1-1]
	\arrow["{(v_F)^*}"', shift right, from=3-1, to=3-3]
	\arrow["{(v_F)_*}"', shift right, from=3-3, to=3-1]
\end{tikzcd}\]
Here, $A_F$ and $A_{j_*F}$ are the compact quotients produced by $v_F$ and $v_{j_*F}$, respectively. The morfism $H\colon A\to A_F$ is defined by $H=v_F\circ j$. Futhermore, $(v_F)_*$ and $(v_{j_*F})_*$ are inclusions.\\

Let $h\colon A_{j_*F}\to A_j$ be such that, for $x\in A_{j_*F}$, $h(x)=H(x)$. Therefore, if $h=H_{\mid{A_{j_*F}}}$, then the above diagram commutes.\\

We need that $h$ to be a frame morphism. First, by the difinition of $h$, this is $\wedge$-morphism. It remains to be seen that $h$ is $\bigvee$-morphism.\\

The joins in $A_{j_*F}$ and $A_F$ are calculated differently. Thus, let $\hat{\bigvee}$ be join in $A_{j_*F}$ and let $\tilde{\bigvee}$ be join in $A_F$. Therefore
\[
\hat{\bigvee}=v_{j_*F}\circ \bigvee\quad\mbox{ and }\quad \tilde{\bigvee}=v_{F}\circ \bigvee,
\] 
that is, for $X\subseteq A$, $Y\subseteq A_j$,
\[
	\hat{\bigvee}X=v_{j_*F}(\bigvee X)\quad\mbox{ and }\quad \tilde{\bigvee}Y=v_{F}(\bigvee Y).
\]

Since $H$ is a frame morphism, then $H\circ \bigvee=\tilde{\bigvee}\circ H$. Let us get something similar to $h$.

\begin{lem}\label{bigvee g}
$h\circ \hat{\bigvee}=\tilde{\bigvee}\circ h$.
\end{lem}

\begin{proof}
It is enough to check the comparison $h\circ \hat{\bigvee}\leq \tilde{\bigvee}\circ h$. Thus
\[
h\circ \hat{\bigvee}=H\circ v_{j_*F}\circ \bigvee=v_F\circ j\circ v_{j_*F}\circ \bigvee\leq v_F\circ v_F\circ j\circ \bigvee
\]
where the inequality is the Corollary \ref{finftyf}. Futhermore, $v_F\circ v_F=v_F$, then
\[
h\circ\hat{\bigvee}\leq v_F\circ j\circ \bigvee =H\circ \bigvee=\tilde{\bigvee}\circ H=\tilde{\bigvee}\circ h.
\]
Therefore $h\circ\hat{\bigvee}=\tilde{\bigvee}\circ h$.
\end{proof}

With this we prove the following.

\begin{prop}\label{VFsquare}
The diagram
\[\begin{tikzcd}
	A & {A_{j_*F}} \\
	{A_j} & {A_F}
	\arrow["{v_{j_*F}}", from=1-1, to=1-2]
	\arrow["j"', from=1-1, to=2-1]
	\arrow["h", from=1-2, to=2-2]
	\arrow["{v_F}"', from=2-1, to=2-2]
\end{tikzcd}\]
is commutative.
\end{prop}

With the above diagram, we could analyze some compact quotients, for example, closed compact quotients.

\begin{dfn}\label{ccquotien}
Let $A$ be a frame and $F\in A^\wedge$. The compact quotient $A_F$ is closed if $A_F=A_{u_d}$ for some $d\in A$.
\end{dfn}
\begin{prop}\label{tidyquout}
    If $A$ is a tidy frame, then $A_j$ is tidy.
\end{prop}

\begin{proof}
It is easy to prove that $F\subseteq j_*F$. Since $A$ is tidy and $F\in A^\wedge$, it is true that 
\[
x\in F\Rightarrow \hat{d}\vee x=1,
\]
where $\hat{d}=d(\alpha)=f^\alpha(0)$.\\
If $\hat{d}\leq d$, then $d\vee x=1$, for $d=d(\alpha)=f^\alpha(j(0))$.\\

Thus, for Corollary \ref{finftyf}
\[
\hat{d}=\hat{d}(\alpha)\leq j(\hat{d}(\alpha))=j(\hat{f}^\alpha(0))\leq f^\alpha(j(0))=d(\alpha)=d.
\]

Therefore if $x\in F$, then $d\vee x=1$ and $A_j$ is tidy.
\end{proof}

\begin{prop}\label{KCquout}
    If $A$ has KC, then $A_j$ has KC for every $j\in N(A).$
\end{prop}

\begin{proof}
We consider $k\in NA_j$ such that $(A_j)_k$ is compact. 
Since any open filter is admissible, we have $\nabla(k)\in A_j^\wedge$ 
and by Proposition \ref{fF} $j_*\nabla(K)\in A^\wedge$.\\

Let $l=j_*\circ  k\circ j^*\in NA$ be, then $A_l$ is a compact quotient of $A$ and exists $a\in A$ such that $l=u_a$. Thus, we have 
\[
\begin{tikzcd}
	A \arrow[r, "j^*"'] \arrow[rrr, "l", bend left] & A_j \arrow[r, "k"'] & (A_j)_k \arrow[r, "j_*"'] & A_j\subseteq A
	\end{tikzcd}\]
and $a\vee x=k(j(x))$. Therefore, if $x=a$, $k(j(x))=a$.\\

We need that $k=u_b$ for some $b\in A_j$. For $x\in A_j$ and $b=j(a)$
\[
\begin{split}
u_b(x)= b\vee x= b\vee j(x)& =j(j(a)\vee j(x))\\
& =j(k(j(a))\vee x)\\
& =j(u_a(x))\\
& =j(k(x))\\	
&=k(x).
\end{split}
\]
Therefore $u_b=k$.
\end{proof}

\begin{prop}\label{KCT1}
If $A$ is a $KC$ frame, the $A$ is a $T_1$ frame.
\end{prop}

\begin{proof}
A frame is $T_1$ if and only if for all $p\in \pt A$, $p$ is maximal. Let $p\in \pt A$ and $a\in A$ be such that $p\leq a\leq 1$. We consider 
\[
w_p(x)=\left\{\begin{array}{lcc}
1 & \mbox{ si } & x\nleq p\\
\\
p & \mbox{ si } & x\leq p
\end{array}\right.
\]
for $x\in A$. $P=\nabla(w_p)=\{x\in A\mid x\nleq p\}$ is a filter completely prime (in particular, $P\in A^\wedge$). Since $A$ is $KC$, then $A_{w_p}$ is a closed compact quotient. Thus $u_p=w_p$, futhermore
\[
u_p(a)=a\quad \mbox{and}\quad w_p(a)=1.
\]
that is, $a=1$. Therefore $p$ is maximum. 
\end{proof}

\section{Admissibility intervals}

The block structure on a frame is an important problem and its related with some separation properties of frames.

\begin{prop}\label{morfismo}
For $F\in A^\wedge$ and $Q\in\mathcal{Q}S$, if $j\in [v_Q, w_Q]$, then $U_*jU^*\in [v_F, w_F]$, where $U^*$ is the morfism spatial reflection $U_*$ is the right adjoint.
\end{prop}

\begin{proof}
Since $N$ is a functor, we have 
\[\begin{tikzcd}
	A & NA \\
	\\
	{\mathcal{O}S} & {N\mathcal{O}S}
	\arrow[""{name=0, anchor=center, inner sep=0}, "U"', from=1-1, to=3-1]
	\arrow[""{name=1, anchor=center, inner sep=0}, "{N(U)}", from=1-2, to=3-2]
	\arrow["{N(\_)}", shorten <=7pt, shorten >=7pt, maps to, from=0, to=1]
\end{tikzcd}\]
and $N(U)_*$ is the right adjoint of $N(U)^\wedge$. Note the following:
\begin{enumerate}
	\item $N(U)(j)\leq k\Leftrightarrow j\leq N(U)_*k$.
	\item If $k\in N\mathcal{O}S$ then $N(U)(j)\leq k\Leftrightarrow Uj\leq kU$.
	\item $N(U)_*k=U_*kU^*$ and $UN(U)_*k=k(U)$.
\end{enumerate}
In 3), if $j=k$, $N(U)_*(j)=U_*jU^*$ and $UN(U)_*j=jU$. For $x\in F$
\[\begin{tikzcd}
	x\in A & {\mathcal{O}S} & {\mathcal{O}S} & A
	\arrow["{U^*}", from=1-1, to=1-2]
	\arrow["j", from=1-2, to=1-3]
	\arrow["{U_*}", from=1-3, to=1-4]
\end{tikzcd}\]
and $U_*(j(U(x))=\bigwedge(S\setminus j(U(x)))$. Note that $U_*(j(U^*(x)))\subseteq \pt A$. Thus

\[
\begin{split}
x\in F \Leftrightarrow Q\subseteq U(x) &\Leftrightarrow U(x)\in \nabla(j)=\nabla(Q)\Leftrightarrow S\setminus j(U(x))=\emptyset\\
& \Leftrightarrow \bigwedge (S\setminus j(U(x)))=1=(U_*jU^*)(x)\\
&\Leftrightarrow x\in \nabla(U_*jU^*)
\end{split}
\]
Therefor $F=\nabla(U_*jU^*)$.
\end{proof}
In this way we have a function 
\[
\mho\colon [V_Q, W_Q]\to [V_F, W_F]
\]

%\begin{prop}\label{Bloqtri}
%For every $A\in \EuScript{Hrm}rm$ the interval corresponding to the block determined by a open filter $F\in A^{\wedge}$ is trivial, that is,\[[v_{F},w_{F}]=\{*\}\]
%
%\end{prop}

%\begin{proof}
%  We know that for all $F\in A^\wedge$ the following holds: $v_F\leq w_F$. As a contradition, suppose that exists $F\in A^\wedge$ such that $w_F\nleq v_F$, that is, exists $a\in A$ such that $w_F(a)\nleq v_F(a)$.\\

%  Note that $w_F(a)\neq 1$, otherwise 
%  \[
%  1=w_F(a)=\bigwedge \{p\in M\mid a\leq p\}\leq p
%  \]
%  and this is a contradition because $p\neq 1$.\\

%Then $1\neq w_F(a)\nleq v_F(a)$ and for the property ($\mathbf{H}$), exists $u\in A$ such that
%\begin{equation}\label{vFwF en a}
%u\nleq w_F(a)\quad \mbox{ and }\quad \neg u \nleq v_F(a)
%\end{equation}

% Due to monotony, $w_F(0)\leq w_F(a)$ and $v_F(0)\leq v_F(a)$- Thus, for (\ref{vFwF en a}) we have that
%\begin{equation}\label{vFwF en 0}
%i)\,u\nleq w_F(0)\quad \mbox{ and }\quad ii)\,\neg u\nleq v_F(0).
%\end{equation}

%For (\ref{vFwF en 0})-$(i)$ is true that $u\nleq \bigwedge M$, in particular, $u\nleq p$ for some $p\in M$. Therefore, $\neg u\leq p$ and $\neg u\leq w_F(0)$. If (\ref{vFwF en 0})-$(ii)$ is true, then $u\notin F$, in otherwise 
%\[
%u\in F\Rightarrow v_u\leq f \Rightarrow v_u(0)=\neg u\leq f(0)
%\] 
%and this is a contradition. Thus, for the Birkhoff's separation Lemma, exists a completely prime filter $G$ such that $u\notin G\supseteq F$. We take
%\[
%q=\bigvee \{y\in A\mid y\notin G\}
%\]
%the point corresponding to $G$. Thus, $u\notin G$, $u\leq q$. If $q\notin F$, then $q\in M$ and $u\nleq q$. Hence $u\leq q$, $u\nleq q$ and this is a contradition.
%\end{proof}

%A consequence of the Proposition \ref{Bloqtri} is that $v_F=w_F$, so that, $A_{v_F}=A_{w_F}$ and $A_{w_F}\simeq \mathcal{O}M\simeq \mathcal{O}Q$. Thus, for all $j\in KA$ we have that $j=v_F$. Then in the Huasdorff case

%\[\begin{tikzcd}
%	A & {A_F} \\
%	{\mathcal{O}S} & {\mathcal{O}S_\nabla}
%	\arrow[from=1-1, to=1-2]
%	\arrow[from=1-1, to=2-1]
%	\arrow["g", from=1-2, to=2-2]
%	\arrow[from=2-1, to=2-2]
%\end{tikzcd}
%\]
%where $g$ (seria el h del \ref{bigvee g})is an isomorphism and $A_F\simeq \mathcal{O}Q\simeq \mathcal{O}S_\nabla$.\\
%HAY que explicar mejor los isorfismos quien es M quien es 
%Q
%On the other hand, $U_*u_{Q'}U^*=v_F$ if and only if $u_{Q'}U^*=U^*v_F$, for the adjuntion properties and $U^*$ the spatial reflection morphism. Therefore

%\[\begin{tikzcd}
%	A & A \\
%	{\mathcal{O}S} & {\mathcal{O}S}
%	\arrow["{v_F}", from=1-1, to=1-2]
%	\arrow["U", shift left=2, from=1-1, to=2-1]
%	\arrow["U"', shift right=2, from=1-2, to=2-2]
%	\arrow["{U_*}"{pos=0.6}, shift left=3, from=2-1, to=1-1]
%	\arrow["{v_\nabla}"', from=2-1, to=2-2]
%	\arrow["{U_*}"', shift right=2, from=2-2, to=1-2]
%\end{tikzcd}\]
%so that, if $A\in \EuScript{Hrm}rm$, then $KC$ implies patch trivial.\\

%The above is the proof of the following theorem.

%\begin{thm}\label{C.Hausdorff}
%If $A\in \EuScript{Hrm}rm$. then every compact quotient is isomrphic to a closed quotient of the topology of a Hausdorff space.
%\end{thm}




%\begin{cor}\label{Viet}
%	If $A\in \EuScript{Hrm}rm$.
%	\[\EuScript{Q}(S)\cong\pt(V(A))\] 
%	\end{cor}

%\begin{prop}\label{Himplies pt}
%Every Hausdorff frame $A$ (in the sense of Johnstone and Shou) is tidy, that is, $A$ is patch trivial. 
%\end{prop}
EJEMPLOS DE marcos ptrivial que no sean KC


	

















\bibliographystyle{amsalpha}
%\begin{thebibliography}{99}
	%\bibitem{D.P.} C. H. Dowker; D. Strauss, \textit{Separation axioms for frames}, Topics in Topology, pp. 223–240. Proc. Colloq., Keszthely, 1972. Colloq. Math. Soc. Janos Bolyai, vol. 8, North-Holland, Amsterdam, 1974.
  
	%\bibitem{Ib.} J. R. Isbell, \textit{Atomless parts of spaces}, Math. Scand. 31 (1972) 5–32.
  
	%\bibitem{P.T.} Johnstone, P. T. (1982). Stone spaces (Vol. 3). Cambridge university press.
  
	%\bibitem{J.S.} P. T. Johnstone; S.-H. Sun, \textit{Weak products and Hausdorff locales}, Categorical Algebra and its Applications, pp. 173–193. Lecture Notes in Mathematics, vol. 1348. Springer-Verlag, Berlin, 1988.
  
	%\bibitem{J.M.} J. Monter; A. Zaldívar, \textit{El enfoque locálico de las reflexiones booleanas: un análisis en la categoría de marcos} [tesis de maestría], 2022. Universidad de Guadalajara.
  
	%\bibitem{P.S.}J. Paseka; B. Šmarda, \textit{T2-frames and almost compact frames}, Czechoslovak Math. J. 42 (1992) 297–313.
	
	%\bibitem{J.P.} Picado, J., Pultr, A. (2011). Frames and Locales: topology without points. Springer Science and Business Media.
	
	%\bibitem{J.P.2} Picado, J., Pultr, A. (2021). Separation in point-free topology. Switzerland: Birkhäuser.
	
	%\bibitem{Ro.S.} J. Rosický; B. Šmarda, \textit{T1-locales}, Math. Proc. Cambridge Philos. Soc. 98 (1985) 81–86.
	
	%\bibitem{R.S.} RA. Sexton, \textit{A point free and point-sensitive analysis of the patch assembly}, The University of Manchester (United Kingdom), 2003.
  
	%\bibitem{R.S.2} Sexton, R. A., Simmons, H. (2006). An ordinal indexed hierarchy of separation properties. In Topology Proc (Vol. 30, No. 2, pp. 585-625).

	%\bibitem{R.S.3} Sexton, R. A., Simmons, H. (2006). Point-sensitive and point-free patch constructions. Journal of Pure and Applied Algebra, 207(2), 433-468.
	  
	%\bibitem{H.S.3} H. Simmons, \textit{The assembly of a frame}, University of Manchester (2006).
  
	%\bibitem{H.W.} H. Wallman, \textit{Lattices and topological spaces}, Ann. Math. 39 (1938) 112–126.
	
	%\bibitem{A.Z.} A. Zaldívar, \textit{Introducción a la teoría de marcos} [notas curso], 2024. Universidad de Guadalajara.

	%\bibitem{H.S.(V)} Simmons, H. (2004). The Vietoris modifications of a frame. Unpublished manuscript, 79pp., available online at http://www. cs. man. ac. uk/hsimmons.

	%\bibitem{W.A.} Wilansky, A. (1967). Between T1 and T2, this MONTHLY.

%\end{thebibliography}
\bibliography{refer} 
%\bibliography{research2}

\end{document}