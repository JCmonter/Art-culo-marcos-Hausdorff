%\documentclass[1p]{elsarticle}
\documentclass[11pt]{amsart}
\usepackage[mathscr]{eucal}
\usepackage{times}
\usepackage{amsmath,amssymb,latexsym,amscd}   
\usepackage{hyperref}
\hypersetup{colorlinks=true,linkcolor=blue,citecolor=red,linktocpage=true}
%\input xypic
%\xyoption{all}
\usepackage{graphicx}
\usepackage[all,cmtip]{xy}
\usepackage{fancyhdr}
\usepackage{mathalfa}
\usepackage{mathrsfs}
%\usepackage[sans]{dsfont}
\usepackage{upgreek}
%\usepackage{mathpazo}
\usepackage{tikz}
\usepackage{graphicx}
\usepackage{tikz-cd}
%%%%%%%%%%%%%%%%%%

\DeclareMathOperator{\op}{op}
\DeclareMathOperator{\spec}{spec}
\DeclareMathOperator{\pt}{pt}
\DeclareMathOperator{\Sp}{sp}
\DeclareMathOperator{\Frm}{Frm}


\theoremstyle{plain}
\newtheorem*{theorem*}{Theorem}
\newtheorem{thm}{Theorem}[section]
\newtheorem{cor}[thm]{Corollary}
\newtheorem{lem}[thm]{Lemma}
\newtheorem{prop}[thm]{Proposition}

\theoremstyle{definition}
\newtheorem{dfn}[thm]{Definition}
\newtheorem{obs}[thm]{Remark}
\newtheorem{ej}[thm]{Example}
  %%
    %\providecommand{\corollaryname}{Corollary}
  %\providecommand{\examplename}{Example}
  %\providecommand{\lemmaname}{Lemma}
  %\providecommand{\propositionname}{Proposition}
  %\providecommand{\remarkname}{Remark}
%\providecommand{\theoremname}{Theorem}
%\providecommand{\subexamplename}{Subexample}
  %\providecommand{\definitionname}{Definition}
\begin{document}
%\begin{frontmatter}

\title{The patch frame and its relations with separation in point-free topology}


\author{}
\email{}
\address{}


\address{}
%$\cortext[cor1]{Corresponding author}

\begin{abstract}

\end{abstract}

\maketitle
\section{Introduction}
Aquí va la introducción.
\section{Preliminaries}\label{pre}





\section{Hausdorff properties implies patch triviality}
Gadgets:

$A$ the base frame

Its point space $S=\pt(A)$.

$NA$ is the assembly of nucleus of $A$.
 
The compact saturated sets of $S$, \[\mathcal{Q}(S).\]

The preframe of open filters of $A$,
\[A^{\wedge}.\]

The preframe of open filters of $\Omega(S)$.

\[\Omega(S)^{\wedge}.\]

The set of compact quotients
\[
KA=\{j\in NA\mid A_j \mbox{ is compact}\}.
\]

First we recall that every open filter $F\in A^{\wedge}$ has three faces, that is, determines (and its determine) by :

\begin{itemize}
\item The compact saturated $Q\in \mathcal{Q}S$.
\item $\nabla\in\Omega(S)^{\wedge}$.
\item The compact quotient $A\rightarrow A_{F}$.
\item The fitted nucleus $v_{F}$.
\end{itemize}
Hoffman-Mislove can be rephrase:

\[\text{There is a bijection between compact quotients of } A \text{ and compact saturated sets of }S\]

\begin{dfn}\label{KOMPACT}
A frame has $\mathrm{KC}$ if every compact quotient of $A$ is a closed one. In other words every compact sublocale is close.
\end{dfn}

Denote by $\EuScript{Hrm}rm$ the subcategory of $Frm$ of Hausdorff frames in the sense of Johnstone and Shu.

\begin{lem}\label{F^}
Let $A\in \Frm$ and $j, k\in NA$ be. We consider $F\in A_j^\wedge$ y $g=j_*kj^*$ where $F=\nabla(k)$. Then $\hat{F}=\nabla(g)\in A^\wedge$.
\end{lem}

\begin{proof}
$\hat{F}$ is a filter, because $g$ is a nucleus. Let us consider $X\subseteq A$ such that $\bigvee X\in \hat{F}$. We must prove that $X\cap \hat{F}\neq \emptyset$.\\

If $\bigvee X\in \hat{F}$, then $g(\bigvee X)=(j_*kj^*)(\bigvee X)=1$. Thus

\[
j^*(\bigvee X)\leq j(\bigvee \{j(x)\mid x\in X\})=j(\bigvee j[X])=\bigvee_j X
\]
and 
\[
1=(j_*k)(j^*(\bigvee X))\leq (j_*k)(\bigvee_j X).
\]

$\{j(x)\mid x\in X\}$ is a directed set because $X$ is directed. Then $j(j[X])\subseteq A_j$, $\bigvee_jX\in F$ and $F\in A_j^\wedge$, so that, exists $x\in X$ such that $x=j(x)\in F$. Therefore $k(j(x))=1$ and $(j_*kj^*)(x)=1$, so that, $x\in \nabla(g)=\hat{F}$. Thus $X\cap \hat{F}\neq \emptyset$.
\end{proof}

\begin{lem}\label{FF^}
For $F\in A^\wedge_j$ and $\hat{F}\in A^\wedge$ as above, then $F\subseteq \hat{F}$.
\end{lem}

\begin{proof}
Let $x\in F$ be, then $j(x)=x$ and $k(x)=1$. Thus 
\[
g(x)=(j_*kj^*)(x)=(j_*k)(j(x))=j_*(k(x))=j_*(1)=1.
\]
Therefore $x\in \hat{F}$.
\end{proof}

\begin{lem}\label{Cociente arreglado}
Let $A\in \Frm$ and $j\in NA$ be. If $A$ is tidy then $A_j$ is tidy.
\end{lem}

The block structure on a frame is an important problem and its related with some separation properties of frames.

\begin{prop}\label{Bloqtri}
For every $A\in \EuScript{Hrm}rm$ the interval corresponding to the block determined by a open filter $F\in A^{\wedge}$ is trivial, that is,\[[v_{F},w_{F}]=\{*\}\]

\end{prop}

\begin{proof}
  We know that for all $F\in A^\wedge$ the following holds: $v_F\leq w_F$. As a contradition, suppose that exists $F\in A^\wedge$ such that $w_F\nleq v_F$, that is, exists $a\in A$ such that $w_F(a)\nleq v_F(a)$.\\

  Note that $w_F(a)\neq 1$, otherwise 
  \[
  1=w_F(a)=\bigwedge \{p\in M\mid a\leq p\}\leq p
  \]
  and this is a contradition because $p\neq 1$.\\

Then $1\neq w_F(a)\nleq v_F(a)$ and for the property ($\mathbf{H}$), exists $u\in A$ such that
\begin{equation}\label{vFwF en a}
u\nleq w_F(a)\quad \mbox{ y }\quad \neg u \nleq v_F(a)
\end{equation}
Note that $0\leq a$, then $w_F(0)\leq w_F(a)$ and $v_F(0)\leq v_F(a)$- Thus, for \ref{vFwF en a} we have that
\begin{equation}\label{vFwF en 0}
i)\,u\nleq w_F(0)\quad \mbox{ y }\quad ii)\,\neg u\nleq v_F(0).
\end{equation}

For \ref{vFwF en 0}-$(i)$ is true that $u\nleq \bigwedge M$, in particular, $u\nleq p$ for all $p\in M$. Therefore, $\neg u\leq p$ and $\neg u\leq w_F(0)$. If \ref{vFwF en 0}-$(ii)$ is true, then $u\notin F$, in otherwise 
\[
u\in F\Rightarrow v_u\leq f \Rightarrow v_u(0)=\neg u\leq f(0)
\] 
and this is a contradition. Thus, for the Birkhoff's separation lemma, exists a completely prime filter $G$ such that $u\notin G\supseteq F$. We take
\[
q=\bigvee \{y\in A\mid y\notin G\}
\]
the point corresponding to $G$. Thus, $u\notin G$, $u\leq q$. If $q\notin F$, then exists $m\in M$ such that $q\leq m$. Sinse $q$ is maximum, $q=m$ or $m=1$, but $m\neq 1$ ($1\in F$ and $M=A\setminus F$), then $m=q\in M$. Hence $u\nleq q$ and this is a contradition. Therefore $v_F=w_F$.
\end{proof}

A consequence of the Proposition \ref{Bloqtri} is that $v_F=w_F$, so that, $A_{v_F}=A_{w_F}$ and $A_{w_F}\simeq \mathcal{O}M\simeq \mathcal{O}Q$. Thus, for all $j\in KA$ we have that $j=v_F$. Then in the Huasdorff case

\[\begin{tikzcd}
	A & {A_F} \\
	{\mathcal{O}S} & {\mathcal{O}S_\nabla}
	\arrow[from=1-1, to=1-2]
	\arrow[from=1-1, to=2-1]
	\arrow["g", from=1-2, to=2-2]
	\arrow[from=2-1, to=2-2]
\end{tikzcd}
\]
where $g$ is an isomorphism and $A_F\simeq \mathcal{O}Q\simeq \mathcal{O}S_\nabla$.\\

On the other hand, $U_*u_{Q'}U^*=v_F$ if and only if $u_{Q'}U^*=U^*v_F$, for the adjuntion properties and $U^*$ the spatial reflection morphism. Therefore

\[\begin{tikzcd}
	A & A \\
	{\mathcal{O}S} & {\mathcal{O}S}
	\arrow["{v_F}", from=1-1, to=1-2]
	\arrow["U", shift left=2, from=1-1, to=2-1]
	\arrow["U"', shift right=2, from=1-2, to=2-2]
	\arrow["{U_*}"{pos=0.6}, shift left=3, from=2-1, to=1-1]
	\arrow["{v_\nabla}"', from=2-1, to=2-2]
	\arrow["{U_*}"', shift right=2, from=2-2, to=1-2]
\end{tikzcd}\]
so that, if $A\in \EuScript{Hrm}rm$ then patch trivial implies $KC$.\\

The above is the proof of the following theorem.

\begin{thm}\label{C.Hausdorff}
If $A\in \EuScript{Hrm}rm$. then every compact quotient is isomrphic to a closed quotient of the topology of a Hausdorff space.
\end{thm}

\begin{prop}\label{Himplies pt}
Every Hausdorff frame $A$ (in the sense of Johnstone and Shou) is tidy, that is, $A$ is patch trivial. 
\end{prop}


\begin{proof}




\end{proof}



\bibliographystyle{amsalpha}

\bibliography{research2}

\end{document}