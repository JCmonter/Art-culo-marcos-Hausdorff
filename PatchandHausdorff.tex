%\documentclass[1p]{elsarticle}
\documentclass[11pt]{amsart}
\usepackage[mathscr]{eucal}
\usepackage{times}
\usepackage{amsmath,amssymb,latexsym,amscd}   
\usepackage{hyperref}
\hypersetup{colorlinks=true,linkcolor=blue,citecolor=red,linktocpage=true}
%\input xypic
%\xyoption{all}
\usepackage{graphicx}
\usepackage[all,cmtip]{xy}
\usepackage{fancyhdr}
\usepackage{mathalfa}
\usepackage{mathrsfs}
%\usepackage[sans]{dsfont}
\usepackage{upgreek}
%\usepackage{mathpazo}
\usepackage{tikz}
\usepackage{graphicx}
\usepackage{tikz-cd}
%%%%%%%%%%%%%%%%%%

\DeclareMathOperator{\op}{op}
\DeclareMathOperator{\spec}{spec}
\DeclareMathOperator{\pt}{pt}
\DeclareMathOperator{\Sp}{sp}
\DeclareMathOperator{\Frm}{Frm}


\theoremstyle{plain}
\newtheorem*{theorem*}{Theorem}
\newtheorem{thm}{Theorem}[section]
\newtheorem{cor}[thm]{Corollary}
\newtheorem{lem}[thm]{Lemma}
\newtheorem{prop}[thm]{Proposition}

\theoremstyle{definition}
\newtheorem{dfn}[thm]{Definition}
\newtheorem{obs}[thm]{Remark}
\newtheorem{ej}[thm]{Example}
  %%
    %\providecommand{\corollaryname}{Corollary}
  %\providecommand{\examplename}{Example}
  %\providecommand{\lemmaname}{Lemma}
  %\providecommand{\propositionname}{Proposition}
  %\providecommand{\remarkname}{Remark}
%\providecommand{\theoremname}{Theorem}
%\providecommand{\subexamplename}{Subexample}
  %\providecommand{\definitionname}{Definition}
\begin{document}
%\begin{frontmatter}

\title{The patch frame and its relations with separation in point-free topology}


\author{}
\email{}
\address{}


\address{}
%$\cortext[cor1]{Corresponding author}

\begin{abstract}

\end{abstract}

\maketitle
\section{Introduction}
Aquí va la introducción.
\section{Preliminaries}\label{pre}





\section{Hausdorff properties implies patch triviality}
Gadgets:

$A$ the base frame

Its point space $S=\pt(A)$.

$NA$ is the assembly of nucleus of $A$.
 
The compact saturated sets of $S$, \[\mathcal{Q}(S).\]

The preframe of open filters of $A$,
\[A^{\wedge}.\]

The preframe of open filters of $\Omega(S)$.

\[\Omega(S)^{\wedge}.\]

The set of compact quotients
\[
KA=\{j\in NA\mid A_j \mbox{ is compact}\}.
\]

First we recall that every open filter $F\in A^{\wedge}$ has three faces, that is, determines (and its determine) by :

\begin{itemize}
\item The compact saturated $Q\in \mathcal{Q}S$.
\item $\nabla\in\Omega(S)^{\wedge}$.
\item The compact quotient $A\rightarrow A_{F}$.
\item The fitted nucleus $v_{F}$.
\end{itemize}
Hoffman-Mislove can be rephrase:

\[\text{There is a bijection between compact quotients of } A \text{ and compact saturated sets of }S\]

Also, if $j^*\colon A\to B$ is a monotone morphism and $F\subseteq A$, $G\subseteq B$ filters in $A$, $B$, respectively, we can produce new filters as follows
\begin{equation}\label{Imagenfiltros}
b\in j^*F \Leftrightarrow j_*(b)\in F\quad \mbox{ and}\quad a\in j_*G \Leftrightarrow j^*(a)\in G
\end{equation}
where $a\in A, b\in B$ and $j_*$ is the right adjoint of $j^*$. Here $j^*F\subseteq B$ and $j_*G\subseteq A$ are filters on $B$ and $A$, respectively.\\

\begin{prop}\label{fF}
For $j=j^*\colon A\to B$ a monotone morphism and $G\in B^\wedge$, then $j_*G\in A^\wedge$.
\end{prop}

\begin{proof}
By (\ref{Imagenfiltros}), $j_*G$ is a filter on $A$. We need $j_*G$ to satisfy the open filter condition. Let $X\subseteq A$ be such that $\bigvee X\in j_*G$, with $X$ directed. Then
\[
Y=\{j(x)\mid x\in X\}
\] 
is directed and $j(\bigvee X)=\bigvee j[X]=\bigvee Y\in G$. Since $G$ is a open filter, exists $y=j(x)\in Y$ such that $y\in G$. Thus $x\in j_*G$, so that, $j_*G\in A^\wedge$.
\end{proof}

\begin{dfn}\label{KOMPACT}
A frame has $\mathrm{KC}$ if every compact quotient of $A$ is a closed one. In other words every compact sublocale is close.
\end{dfn}

Denote by $\EuScript{Hrm}rm$ the subcategory of $Frm$ of Hausdorff frames in the sense of Johnstone and Shu.\\

In (citar el artículo) Sexton says that $A\in \mathbf{Frm}$ is \emph{tidy} if for all $F\in A^\wedge$
\[
x\in F\Rightarrow u_d(x)=d\vee x=1
\]
where $d=d(\alpha)=f^\alpha(0)$, $f=\dot{\bigvee}\{v_y\mid y\in F\}$, $v_y\in NA$ and $0=0_A$ (the reason for the last two clarifications will be understood later).\\

We want translate this same notion, but for $A_j$ when $j\in NA$, so that, for all $F\in A_j^\wedge$, if $x\in F$ then $d\vee x=1$, with $d$ similar to before, because for this case we have that $v_y\in NA_j$ and $0_{A_j}=j(0)$.\\

Let's note that we need open filters $F$ in $A$ and in $A_j$, prenuclei $f$ in $A$ and $A_j$ and elements $d$ in $A$ and in $A_j$. To make writing easier, we will denote by $\hat{F}$, $\hat{f}$ and $\hat{d}$, to the open filter, prenuclei and element associated with the prenucleus, respectively, in the frame $A$. In the frame $A_j$ we use the usual notation, so that, $F$, $f$ and $d$ for the open filter, prenuclei and element associated to the prenucleus.\\

Simmons proves in (citar el Vietoris y los resultados) that the diagram
\[\begin{tikzcd}
	A & A \\
	{\mathcal{O}S} & {\mathcal{O}S}
	\arrow["{f^\infty}", from=1-1, to=1-2]
	\arrow["{U_A}"', from=1-1, to=2-1]
	\arrow["{U_A}", from=1-2, to=2-2]
	\arrow["{F^\infty}"', from=2-1, to=2-2]
\end{tikzcd}\]
commutes laxly, so that, $U_A\circ f^\infty\leq F^\infty \circ U_A$.\\

In this diagram $U_A$ is the spatial reflection morphism, $f^\infty$ and $F^\infty$ represent the associated nuclei asociados to the filters $F\in A^\wedge$ and $\nabla\in \mathcal{O}S^\wedge$.\\

We prove something more general here, since we consider the diagram
\[\begin{tikzcd}
	A & A \\
	{A_j} & {A_j}
	\arrow["{\hat{f}^\infty}", from=1-1, to=1-2]
	\arrow["j"', from=1-1, to=2-1]
	\arrow["j", from=1-2, to=2-2]
	\arrow["{f^\infty}"', from=2-1, to=2-2]
\end{tikzcd}\]

\begin{lem}
    For $j$, $f$ and $\hat{f}$ as before, it holds that $j\circ \hat{f}\leq f\circ j$.
\end{lem}
\begin{proof}\label{f1f}
    By (\ref{Imagenfiltros}) is true that
    \[
    \hat{f}=\dot{\bigvee}\{v_y\mid y\in j_*F\}\quad  \mbox{ and } \quad f=\dot{\bigvee}\{v_{j(y)}\mid j(y)\in F\}. 
    \]
then, for $a\in A$ it is hold
\[
v_y(a)=(y\succ a)\leq \hat{f}(a)\leq j(\hat{f}(a)).
\]

Also, for all $a, y\in A$, $(y\succ a)\wedge y=y\wedge a$ and
\[
\begin{split}
j((y\succ a)\wedge y)\leq j(a) & \Leftrightarrow j(y\succ a)\wedge j(y)\leq j(a)\\
& \Leftrightarrow j(y\succ a)\leq (j(y)\succ j(a)).
\end{split}
\]

Thus 
\[
v_y(a)\leq j(\hat{f}(a))\leq (j(y)\succ j(a))=v_{j(y)}(j(a))\leq f(j(a)).
\]

Therefore $j\circ \hat{f}\leq f\circ j$.
\end{proof}

For $\hat{f}$ and $f$ be nuclei, we need their idempotent closeds.

\begin{cor}\label{finftyf}
    For $j$, $f$ and $\hat{f}$ as before, it is hold that $j\circ \hat{f}^\infty\leq f^\infty\circ j$
\end{cor}

\begin{proof}
    For an ordinal $\alpha$ we will check that $j\circ \hat{f}^\alpha\leq f^\alpha\circ j$. We will do it by transfinite induction.\\

    If $\alpha=0$, it is trivial.\\

    For the induction step, we assume that for $\alpha$ it holds. Then
    \[
    j\circ \hat{f}^{\alpha+1}=j\circ \hat{f}\circ \hat{f}^{\alpha}\leq  f\circ j\circ \hat{f}^\alpha\leq f\circ f^\alpha\circ j=f^{\alpha+1}\circ j,
    \]
    where the first inequality is Lemma \ref{f1f} and the second is true by the induction hypothesis.\\

    If$\lambda$ is a limit ordinal, then (HAZLO)
\end{proof}

We now have the tools to prove the following:

\begin{prop}
    If $A$ is a tidy frame, then $A_j$ is tidy.
\end{prop}

\begin{proof}
It is easy to prove that $F\subseteq j_*F$. Since $A$ is tidy and $F\in A^\wedge$, it is true that 
\[
x\in F\Rightarrow \hat{d}\vee x=1,
\]
where $\hat{d}=d(\alpha)=f^\alpha(0)$.\\
If $\hat{d}\leq d$, then $d\vee x=1$, for $d=d(\alpha)=f^\alpha(j(0))$.\\

Thus, for Corollary \ref{finftyf}
\[
\hat{d}=\hat{d}(\alpha)\leq j(\hat{d}(\alpha))=j(\hat{f}^\alpha(0))\leq f^\alpha(j(0))=d(\alpha)=d.
\]

Therefore if $x\in F$, then $d\vee x=1$ and $A_j$ is tidy.
\end{proof}

The block structure on a frame is an important problem and its related with some separation properties of frames.

\begin{prop}\label{morfismo}
For $F\in A^\wedge$ and $Q\in\mathcal{Q}S$, if $j\in [v_Q, w_Q]$, then $U_*jU^*\in [v_F, w_F]$, where $U^*$ is the morfism spatial reflection $U_*$ is the right adjoint.
\end{prop}

\begin{proof}
Since $N$ is a functor, we have 
\[\begin{tikzcd}
	A & NA \\
	\\
	{\mathcal{O}S} & {N\mathcal{O}S}
	\arrow[""{name=0, anchor=center, inner sep=0}, "U"', from=1-1, to=3-1]
	\arrow[""{name=1, anchor=center, inner sep=0}, "{N(U)}", from=1-2, to=3-2]
	\arrow["{N(\_)}", shorten <=7pt, shorten >=7pt, maps to, from=0, to=1]
\end{tikzcd}\]
and $N(U)_*$ is the right adjoint of $N(U)^\wedge$. Note the following:
\begin{enumerate}
	\item $N(U)(j)\leq k\Leftrightarrow j\leq N(U)_*k$.
	\item If $k\in N\mathcal{O}S$ then $N(U)(j)\leq k\Leftrightarrow Uj\leq kU$.
	\item $N(U)_*k=U_*kU^*$ and $UN(U)_*k=k(U)$.
\end{enumerate}
In 3), if $j=k$, $N(U)_*(j)=U_*jU^*$ and $UN(U)_*j=jU$. For $x\in F$
\[\begin{tikzcd}
	x\in A & {\mathcal{O}S} & {\mathcal{O}S} & A
	\arrow["{U^*}", from=1-1, to=1-2]
	\arrow["j", from=1-2, to=1-3]
	\arrow["{U_*}", from=1-3, to=1-4]
\end{tikzcd}\]
and $U_*(j(U(x))=\bigwedge(S\setminus j(U(x)))$. Note that $U_*(j(U^*(x)))\subseteq \pt A$. Thus

\[
\begin{split}
x\in F \Leftrightarrow Q\subseteq U(x) &\Leftrightarrow U(x)\in \nabla(j)=\nabla(Q)\Leftrightarrow S\setminus j(U(x))=\emptyset\\
& \Leftrightarrow \bigwedge (S\setminus j(U(x)))=1=(U_*jU^*)(x)\\
&\Leftrightarrow x\in \nabla(U_*jU^*)
\end{split}
\]
Therefor $F=\nabla(U_*jU^*)$.
\end{proof}
In this way we have a function 
\[
\mho\colon [V_Q, W_Q]\to [V_F, W_F]
\]

\begin{prop}\label{Bloqtri}
For every $A\in \EuScript{Hrm}rm$ the interval corresponding to the block determined by a open filter $F\in A^{\wedge}$ is trivial, that is,\[[v_{F},w_{F}]=\{*\}\]

\end{prop}

\begin{proof}
  We know that for all $F\in A^\wedge$ the following holds: $v_F\leq w_F$. As a contradition, suppose that exists $F\in A^\wedge$ such that $w_F\nleq v_F$, that is, exists $a\in A$ such that $w_F(a)\nleq v_F(a)$.\\

  Note that $w_F(a)\neq 1$, otherwise 
  \[
  1=w_F(a)=\bigwedge \{p\in M\mid a\leq p\}\leq p
  \]
  and this is a contradition because $p\neq 1$.\\

Then $1\neq w_F(a)\nleq v_F(a)$ and for the property ($\mathbf{H}$), exists $u\in A$ such that
\begin{equation}\label{vFwF en a}
u\nleq w_F(a)\quad \mbox{ and }\quad \neg u \nleq v_F(a)
\end{equation}

 Due to monotony, $w_F(0)\leq w_F(a)$ and $v_F(0)\leq v_F(a)$- Thus, for \ref{vFwF en a} we have that
\begin{equation}\label{vFwF en 0}
i)\,u\nleq w_F(0)\quad \mbox{ and }\quad ii)\,\neg u\nleq v_F(0).
\end{equation}

For \ref{vFwF en 0}-$(i)$ is true that $u\nleq \bigwedge M$, in particular, $u\nleq p$ for all $p\in M$. Therefore, $\neg u\leq p$ and $\neg u\leq w_F(0)$. If \ref{vFwF en 0}-$(ii)$ is true, then $u\notin F$, in otherwise 
\[
u\in F\Rightarrow v_u\leq f \Rightarrow v_u(0)=\neg u\leq f(0)
\] 
and this is a contradition. Thus, for the Birkhoff's separation lemma, exists a completely prime filter $G$ such that $u\notin G\supseteq F$. We take
\[
q=\bigvee \{y\in A\mid y\notin G\}
\]
the point corresponding to $G$. Thus, $u\notin G$, $u\leq q$. If $q\notin F$, then $q\in M$ and $u\nleq q$. Hence $u\leq q$, $u\nleq q$ and this is a contradition.
\end{proof}

A consequence of the Proposition \ref{Bloqtri} is that $v_F=w_F$, so that, $A_{v_F}=A_{w_F}$ and $A_{w_F}\simeq \mathcal{O}M\simeq \mathcal{O}Q$. Thus, for all $j\in KA$ we have that $j=v_F$. Then in the Huasdorff case

\[\begin{tikzcd}
	A & {A_F} \\
	{\mathcal{O}S} & {\mathcal{O}S_\nabla}
	\arrow[from=1-1, to=1-2]
	\arrow[from=1-1, to=2-1]
	\arrow["g", from=1-2, to=2-2]
	\arrow[from=2-1, to=2-2]
\end{tikzcd}
\]
where $g$ is an isomorphism and $A_F\simeq \mathcal{O}Q\simeq \mathcal{O}S_\nabla$.\\

On the other hand, $U_*u_{Q'}U^*=v_F$ if and only if $u_{Q'}U^*=U^*v_F$, for the adjuntion properties and $U^*$ the spatial reflection morphism. Therefore

\[\begin{tikzcd}
	A & A \\
	{\mathcal{O}S} & {\mathcal{O}S}
	\arrow["{v_F}", from=1-1, to=1-2]
	\arrow["U", shift left=2, from=1-1, to=2-1]
	\arrow["U"', shift right=2, from=1-2, to=2-2]
	\arrow["{U_*}"{pos=0.6}, shift left=3, from=2-1, to=1-1]
	\arrow["{v_\nabla}"', from=2-1, to=2-2]
	\arrow["{U_*}"', shift right=2, from=2-2, to=1-2]
\end{tikzcd}\]
so that, if $A\in \EuScript{Hrm}rm$ then patch trivial implies $KC$.\\

The above is the proof of the following theorem.

\begin{thm}\label{C.Hausdorff}
If $A\in \EuScript{Hrm}rm$. then every compact quotient is isomrphic to a closed quotient of the topology of a Hausdorff space.
\end{thm}

\begin{prop}\label{Himplies pt}
Every Hausdorff frame $A$ (in the sense of Johnstone and Shou) is tidy, that is, $A$ is patch trivial. 
\end{prop}


\begin{proof}




\end{proof}



\bibliographystyle{amsalpha}

\bibliography{research2}

\end{document}