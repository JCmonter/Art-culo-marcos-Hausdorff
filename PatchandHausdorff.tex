%\documentclass[1p]{elsarticle}
\documentclass[11pt]{amsart}
\usepackage[mathscr]{eucal}
\usepackage{times}
\usepackage{amsmath,amssymb,latexsym,amscd}   
\usepackage{hyperref}
\hypersetup{colorlinks=true,linkcolor=blue,citecolor=red,linktocpage=true}
%\input xypic
%\xyoption{all}
\usepackage{graphicx}
\usepackage[all,cmtip]{xy}
\usepackage{fancyhdr}
\usepackage{mathalfa}
\usepackage{mathrsfs}
%\usepackage[sans]{dsfont}
\usepackage{upgreek}
%\usepackage{mathpazo}
\usepackage{tikz}
\usepackage{graphicx}
\usepackage{tikz-cd}
%%%%%%%%%%%%%%%%%%

\DeclareMathOperator{\op}{op}
\DeclareMathOperator{\spec}{spec}
\DeclareMathOperator{\pt}{pt}
\DeclareMathOperator{\Sp}{sp}



\theoremstyle{plain}
\newtheorem*{theorem*}{Theorem}
\newtheorem{thm}{Theorem}[section]
\newtheorem{cor}[thm]{Corollary}
\newtheorem{lem}[thm]{Lemma}
\newtheorem{prop}[thm]{Proposition}

\theoremstyle{definition}
\newtheorem{dfn}[thm]{Definition}
\newtheorem{obs}[thm]{Remark}
\newtheorem{ej}[thm]{Example}
  %%
    %\providecommand{\corollaryname}{Corollary}
  %\providecommand{\examplename}{Example}
  %\providecommand{\lemmaname}{Lemma}
  %\providecommand{\propositionname}{Proposition}
  %\providecommand{\remarkname}{Remark}
%\providecommand{\theoremname}{Theorem}
%\providecommand{\subexamplename}{Subexample}
  %\providecommand{\definitionname}{Definition}
\begin{document}
%\begin{frontmatter}

\title{The patch frame and its relations with separation in point-free topology}


\author{}
\email{}
\address{}


\address{}
%$\cortext[cor1]{Corresponding author}

\begin{abstract}

\end{abstract}

\maketitle
\section{Introduction}
Aquí va la introducción.
\section{Preliminaries}\label{pre}





\section{Hausdorff properties implies patch triviality}
Gadgets:

$A$ the base frame

Its point space $S=\pt(A)$.

The compact saturated sets of $S$, \[\mathcal{Q}(S).\]

The preframe of open filters of $A$,
\[A^{\wedge}.\]

The preframe of open filters of $\Omega(S)$.

\[\Omega(S)^{\wedge}.\]

First we recall that every open filter $F\in A^{\wedge}$ has three faces, that is, determines (and its determine) by :

\begin{itemize}
\item The compact saturated $Q\in \mathcal{Q}S$.
\item $\nabla\in\Omega(S)^{\wedge}$.
\item The compact quotient $A\rightarrow A_{F}$.
\item The fitted nucleus $v_{F}$.
\end{itemize}
Hoffman-Mislove can be rephrase:

\[\text{There is a bijection between compact quotients of } A \text{and compact saturated sets of }S\]
\begin{dfn}\label{KOMPACT}
A frame has $\mathrm{KC}$ if every compact quotient of $A$ is a closed one. In other words every compact sublocale is close.
\end{dfn}

Denote by $\EuScript{Hrm}rm$ the subcategory of $Frm$ of Hausdorff frames in the sense of Johnstone and Shu.

The block structure on a frame is an important problem and its related with some separation properties of frames.

\begin{prop}\label{Bloqtri}
For every $A\in \EuScript{Hrm}rm$ the interval corresponding to the block determined by a open filter $F\in A^{\wedge}$ is trivial, that is,\[[v_{F},w_{F}]=\{*\}\]

\end{prop}

\begin{proof}
  We know that for all $F\in A^\wedge$ the following holds: $v_F\leq w_F$. As a contradition, suppose that exists $F\in A^\wedge$ such that $w_F\nleq v_F$, that is, exists $a\in A$ such that $w_F(a)\nleq v_F(a)$.\\

  Note that $w_F(a)\neq 1$, otherwise 
  \[
  1=w_F(a)=\bigwedge \{p\in M\mid a\leq p\}\leq p
  \]
  and this is a contradition because $p\neq 1$.\\

Then $1\neq w_F(a)\nleq v_F(a)$ and for the property ($\mathbf{H}$), exists $u\in A$ such that
\begin{equation}\label{vFwF en a}
u\nleq w_F(a)\quad \mbox{ y }\quad \neg u \nleq v_F(a)
\end{equation}
Note that \ref{vFwF en a} is true for all $a\in A$, in particular we have that
\begin{equation}\label{vFwF en 0}
i)\,u\nleq w_F(0)\quad \mbox{ y }\quad ii)\,\neg u\nleq v_F(0).
\end{equation}

For \ref{vFwF en 0}-$(i)$ is true that $u\nleq \bigwedge M$, in particular, $u\nleq p$ for all $p\in M$. Therefore, $\neg u\leq p$ and $\neg u\leq w_F(0)$. If \ref{vFwF en 0}-$(ii)$ is true, then $u\notin F$, in otherwise 
\[
u\in F\Rightarrow v_u\leq f \Rightarrow v_u(0)=\neg u\leq f(0)
\] 
and this is a contradition. Thus, for the Birkhoff's separation lemma, exists a completely prime filter $G$ such that $u\notin G\supseteq F$. We take
\[
q=\bigvee \{y\in A\mid y\notin G\}
\]
the point corresponding to $G$. Thus, $u\notin G$, $u\leq q$. If $q\notin F$, then exists $m\in M$ such that $q\leq m$. Sinse $q$ is maximum, $q=m$ or $m=1$, but $m\neq 1$ ($1\in F$ and $M=A\setminus F$), then $m=q\in M$. Hence $u\nleq q$ and this is a contradition. Therefore $v_F=w_F$.
\end{proof}




\begin{prop}\label{Himplies pt}
Every Hausdorff frame $A$ (in the sense of Johnstone and Shou) is tidy, that is, $A$ is patch trivial. 
\end{prop}


\begin{proof}




\end{proof}



\bibliographystyle{amsalpha}

\bibliography{research2}

\end{document}